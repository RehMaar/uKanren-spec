\documentclass[submission,copyright,creativecommons]{eptcs}
\providecommand{\event}{TEASE-LP 2020} % Name of the event you are submitting to
\usepackage{underscore}                % Only needed if you use pdflatex.

\newcommand{\miniKanren}{\textsc{miniKanren}\ }

\title{Supercompilation Strategies of Relational Progams}
\author{Maria Kuklina
\institute{ITMO Univeristy\\ Saint Petersburg, Russia}
\email{kuklina.md@gmail.com}
\and
Ekaterina Verbitskaia
\institute{Saint Petersburg State Univeristy\\
Saint Petersburg, Russia}
\email{kajigor@gmail.com}
}
\def\titlerunning{A Longtitled Paper}
\def\authorrunning{M. Kuklina, E. Verbitskaia}

\begin{document}
\maketitle

\begin{abstract}
% В этой министатье мы показываем, что мы адаптировали суперкомпиляцию
% (с привкусом CPD) к miniKanren'у, что существуют разные способы
% выбрать конъюнкт для раскрытия и что они дают вот такие результаты для
% наших вот таких вот тестов.

In our work we research methods of supercompilation\cite{introSC} in the context of relational program specialization.
We implemented a supercompiler for \miniKanren and propose different supercompilation strategies.

\end{abstract}

\section{Introduction}

{\it Relational programming} is a pure form of logic programming in which programs are
written as {\it relations}. In relations ``input'' and ``output'' arguments are indistiguishable,
therefore relational programs solve problems in general.

Interesting application of relational programming is {\it relational interpreters}.
Besides computing the output form an input or running a program in an opposite ``direction'',
relational interpreters are capable to both verify a solution and search for it.\cite{lozov2019}

\miniKanren is a family of domain-specific languages specially designed for relational programming.\cite{friedmanSchemer}
Relational interpreters implemented in \miniKanren show all their potential, however, in the context of a particular
task computation performance can be highly insufficient.

Specialization is a technique of automatic program optimization. A {\it specializer} takes a program and a part
of its input and produces a new program that behave the same way on the rest of its input as the original
one on all of its input.\cite{jones1993partial} 
A specializer (in form of {\it conjunctive partial deduction, CPD}) has been implemented and applied to
relational programs in \miniKanren in \cite{lozov2019}. Despite the fact that CPD gives a huge performance boost
comparing to original programs, specialized programs still carries some overhead of interperation.

\section{Supercompilation Strategies}

% Суперкомиляция это вот такая вот хреновина.
Supercompilation is a method of program transformation. Supercompiler tries to symbolically
execute a program for a given {\it configuration} -- an expression with free variables --
tracing a computation history with a {\it graph of configurations} and build an equivalent {\it residual} program removing redundant computation.

More formally, supercompilation's steps are following:
\begin{itemize}
\item {\bf Driving} is a process of symbolic execution with resulting possibly infinite tree of configurations.
\item The goal of {\bf folding} is to avoid building infinite tree by turning it into a finite graph, from which
      the original infinite tree could be recovered. Usually it's done by adding a link to an ancestor if it's a
      renaming of handled configuration.

\item {\bf Generalization} is another way of avoiding an infinite tree, when no folding operations can be done.
     The aim of this step is to generate new goals which can be folded in finite time. Generalization step is applied
     only when a {\bf whisle} decides it's neccessary.

\item {\bf Residualization} is a process of generating an actual program from a graph of configurations.
\end{itemize}

\subsection{Supercompilation for \miniKanren}

We implemented a supercompiler for \miniKanren using a {\it homeomorphic embedding} as a whisle and CPD-like abstraction algorithm
for generalization. However, various strategies could be applied for a driving step.

A driving step of \miniKanren supercompiler handles a configuration which takes a form of a conjunction of calls.
For further computations a supercompiler has to decide which of the conjuncts to unfold.

We implemented several strategies:
\begin{itemize}
\item {\bf Full unfold} strategy unfolding all conjuncts simulteniously.
      However, supercompilation time with full unfold strategy takes signigicant amount of resourcses, so
      we didn't shows a results of it for some tests.
\item {\bf First unfold} strategy always unfolds first conjunct.
\item {\bf Sequential unfold} strategy unfolds conjuncts in order.
\item {\bf Recursive unfold} strategy firstly unfolds conjuncts which have at least one recursive call.
\item {\bf Non-recursive unfold} strategy firstly unfolds conjuncts which don't have any recursive call.
\item {\bf Maximal size unfold} strategy firstly unfold conjuncts with the largest amount of conjunctions (in CNF).
\item {\bf Maximal size unfold} strategy firstly unfold conjuncts with the least amount of conjunctions (in CNF).
\end{itemize}

% % Вот такие вот результаты на данных момент.
\subsection{Results}

We present testing results for the implemented supercompiler with described strategies and comparison with
original interpreter and CPD specializer.  As a specific implementation of \miniKanren we use {\it OCanren} \cite{ocanren};
the supercompiler is written in \textsc{Haskell}.

First, we compare the performance of the solvers for path searching problem. We ran the search on a complete graph $K_{10}$,
searching for path of lengths 9, 11, 13 and 15. The results are presented in \ref{tab:path}.

Second, we comapre the performance of generating of propositional logic formulas. We ran the search for 1000 formulas
in empty substitution and in substitution with only one free variable. The results are presented in \ref{tab:logic}

\begin{table}[ht]

% #Len 9
% FstU : 0.014185s
% #Len 11
% FstU : 0.019156s
% #Len 13
% FstU : 0.021285s
% #Len 15
% FstU : 0.026728s

\parbox{.5\textwidth} {
\begin{tabular}{|c|c|c|c|c|}
\hline
Path length & 9 & 11 & 13 & 15 \\ \hline
\hline
Orignal    & 0.606s & 3.98s & 22.73s & 120.48s \\
CPD        & 0.366s & 2.27s & 12.55s & 63.12s \\ \hline
Full       & 0.021s & 0.03s & 0.035s & 0.041s \\
First      & 0.014s & 0.02s & 0.021s & 0.025s \\
Sequential & 0.014s & 0.02s & 0.022s & 0.027s \\
MaxU       & 0.014s & 0.02s & 0.022s & 0.026s \\
MinU       & 0.014s & 0.02s & 0.022s & 0.027s \\
RecU       & 0.018s & 0.02s & 0.021s & 0.027s \\
NrcU       & 0.014s & 0.02s & 0.022s & 0.026s \\
\hline
\end{tabular}
\label{tab:paths}
\caption{Searching for paths in the $K_{10}$ graph}
}
%\end{table}
%% 
%\begin{table}[ht]
\quad
\quad
\parbox{.4\textwidth}{
\begin{tabular}{|p{3cm}|l|l|}
\hline
Free variables in substitution & 0 free vars & 1 free var  \\ \hline
\hline
Orignal       & 0.19s  & 0.28s \\
CPD           & 1.89s  & 3.33s \\
\hline
First         & 0.216s & 0.15s \\
Sequential    & 0.150s & 0.28s \\
Non recursive & 0.050s & 0.07s \\
Recursive     & 0.045s & 0.45s \\
Maximal size  & 0.136s & 0.19s \\
Minimal size  & 0.046s & 0.06s \\
\hline
\end{tabular}
\label{tab:logic}
\caption{Searching for formulas in a given substitution}
}
\end{table}

%\nocite{*}
\bibliographystyle{eptcs}
\bibliography{main}

\end{document}
