\documentclass[14pt, a4paper] {ncc}
\usepackage[utf8] {inputenc}
\usepackage[T2A]{fontenc}
\usepackage[english, russian] {babel}
\usepackage[top=20mm, bottom=20mm, left=25mm, right=10mm]{geometry}
\usepackage{xcolor}
\usepackage{tikz}

\usepackage{biblatex}

% Для вставки фигур
\usepackage{float}
\usepackage{caption}

\usepackage{longtable,amsmath,amsfonts}

\usepackage{listings}
\usepackage{algpseudocode}
% More Math Symbols
\usepackage{stmaryrd}

% Enumeration
\usepackage{enumitem}
\setlist[enumerate]{topsep=0pt,itemsep=0ex,partopsep=1ex,parsep=1ex}
\setlist[itemize]{itemsep=0ex}

% Полуторный межстрочный интервал
\usepackage[nodisplayskipstretch]{setspace}
\onehalfspacing

% Добавить абзацный отступ для первых абзацев в section/subsection,
% по умолчанию не добавляется
\usepackage{indentfirst}

% Times New Roman
\usepackage{pscyr}
\renewcommand{\rmdefault}{ftm}


% Абзацный отступ равен 1.25 см
\parindent=1.25cm

% Номер страницы по середине верхнего поля
\usepackage{fancyhdr}
\pagestyle{fancy}
\fancyhf{}
\fancyhead[C]{\thepage}
\renewcommand{\headrulewidth}{0pt}


%%%%%%%%%%%%%%%%%%%%%%%%%
\lstdefinelanguage{MyLang}
{
  morekeywords={data},
  keywordstyle=\bfseries\color{black}
}
%%%%%%%%%%%%%%%%%%%%%%%%%
\newcommand{\larrow}{\leftarrow}
\newcommand{\rarrow}{\rightarrow}

\newcommand{\ukanren}{$\mu$Kanren }

\newcommand{\origin}[1]{(англ. {\it #1})}

\newcommand{\todo}[1]{{\bf\color{red}TODO: #1}}
%%%%%%%%%%%%%%%%%%%%%%%%%


\addbibresource{main.bib}

\begin{document}
\setcounter{figure}{0}

%\tableofcontents
%\newpage
%\input{src/sec_intro.tex}
%\newpage
\section{Описание предметной области}

\subsection{Логическое и реляционное программирование}

{\bf Логическое программирование}~--- это вид декларативного программирования,
основанный на логике предикатов первого порядка в форме дизъюнктов Хорна,
применяющий принципы логического вывода на основе заданных фактов и правил вывода.
Программа, написанная на логическом языке --- это множество логических формул,
выражающих факты и правила, описывающих определённую область проблем.
\cite{logicMJ}

Существует множество языков логического программирования, таких как Prolog, Curry, Mercurry,
однако самые известные --- языки семейства Prolog. Prolog применяется для доказательства
теорем, проектирования баз знаний, создания экспертных систем и искусственного интеллекта.

Prolog построен на \emph{методе резолюций}, который является обобщением метода
``доказательства от противного'', а в частности --- на \emph{линейном} методе
резолюций \origin{Linear resolution with Selection function for Definition clauses}.
При вычислении программы правило резолюции применяется не к случайных дизъюнктам,
а в строго установленном порядке. В случае, когда вычисления дизъюнкта прошло
неудачно, происходит \emph{откат} к прошлому состоянию программы, на котором
выбирался неудавшийся дизъюнкт\cite{logicMJ}.
Помимо этого, Prolog вводит разнообразные синтаксическое конструкции с \emph{эффектами},
то есть с действиями, приводящими к изменению окружения программы,
к примеру, оперетор отсечения~\origin{cut}, который влияет на откаты и переходы к другим дизъюнктам.

Описанные выше вещи определяют Prolog, однако из-за них теряется важное свойство,
речь о котором пойдёт ниже.

% 
% Пример программы на Prolog приведён в Листинге~\ref{lst:memberProlog}.
% 
% \begin{lstlisting}[language=Prolog,caption={Проверка принадлежности элемента списку},captionpos=b,label={lst:memberProlog}]
% member(X, [X | T]).
% member(X, [H | T]) :- member(X, T).
% \end{lstlisting}
% 
% В этой программе проверяется принадлежность элемента списку. Есть два возможных
% варианта происходящего: либо элемент равен элементу в голове списка



{\bf Реляционное программирование} --- это форма чистого логического программирования,
в котором программы задаются как набор математических {\it отношений}.

К примеру, сложение $X + Y = Z$ в терминах реляционного программирования может
быть выражено отношением (символ $o$ традиционно используется для обозначения отношения)
\[ \text{add}^o(X, Y, Z), \]
которое в зависимости от того, какие переменные заданы, порождает
все возможные значения переменных, при которых отношение выполняется:
\begin{itemize}
\item $\text{add}^o(1, 2, 3)$ --- проверка выполнимости отношения;
\item $\text{add}^o(1, 2, A)$ --- поиск всех таких A, при которых $1 + 2 = A$;
\item $\text{add}^o(A, B, 3)$ --- поиск всех таких A и B, при которых $A + B = 3$;
\item $\text{add}^o(A, B, C)$ --- поиск всех троек A, B и С, при которых $A + B = C$.
\end{itemize}

Отношения не предполагают функциональных зависимостей между переменными, поэтому
поиск можно проводить в разных ``направлениях'', в зависимости от того, какие переменные
заданы. Когда же чистые отношения вырождаются в функциональные, появляется
явная зависимость в между переменными; тогда можно говорить про запуск
в ``прямом'' направлении (то есть задаются входные аргументы) и в ``обратном''
(при задании результата).

\todo{применение}

% Ссылка на Булычева
%Одно из применений реляционной парадигмы --- {\it реляционные интерпретаторы}.
%Для языка $L$ его интерпретатор -- это функция $\text{eval}_L$, которая принимает
%на вход программу $p_L$ на этом языке, её вход $i$ и возвращает некоторый выход $o$:
%\[ \text{eval}_L (p_L, i) = \llbracket p_L \rrbracket (i) = o \]
%%Реляционную версию интерпретатора можно представить как отношение:
%%\[ \text{eval}_L^o(p_L, i, o). \]
%
%При запуске такого отношения на разном наборе аргументов можно добиться интересных эффектов:
%по программе $p_L$ и выходу $o$ искать возможные входы $i$,
%решать задачи поиска по задаче распознавания~\cite{lozov}, генерировать программы
%по заданной спецификации входа $i$ и выхода $o$ (Programming By Example,~\cite{unifiedMK}).


\subsection{miniKanren}

% Byrd
{\it miniKanren} -- семейство встраеваемых предметно-ориентированных языков, специально спроектированное для
реляционного программирования~\cite{byrdMK}.

Основная реализация miniKanren написана на языке Scheme~\cite{reasonedSchemer}, однако существует множество встраиваний
в ряд других языков, в том числе Clojure, Racket, OCaml, Haskell и другие.

miniKanren предоставляет набор базовых конструкций: унификация ($\equiv$),
конъюнкция $(\land)$, дизъюнкция $(\lor)$, введение свежей переменной (fresh), вызов реляционного отношения,
--- представляющий ядро языка, и разнообразные расширения, к примеру, оператор неэквивалентности
\origin{disequality constraint} или нечистые операторы, предоставляющие функциональность
отсечения из Prolog.

Классический пример --- программа конкатенации двух списков --- указан на рисунке~\ref{fig:appendo}.

\begin{figure}[h!]
\begin{lstlisting}[mathescape,language=Haskell,extendedchars=\true,frame=single,basicstyle=\ttfamily]
$\text{append}^o$ X Y R =
  X $\equiv$ [] $\land$ Y $\equiv$ R $\lor$
  fresh (H X' R')
    (X $\equiv$ H :: X') $\land$
    (R $\equiv$ H :: R') $\land$
    $\text{appendo}^o$ X' Y R'
\end{lstlisting}

\caption{Пример программы на miniKanren (\lstinline{::} --- конструктор списка)}
\label{fig:appendo}
\end{figure}

Пояснение к программе:
список R является конкатенацией списков X и Y в случае, когда список X пуст, а Y равен R, либо
когда X и R раскладываются на голову и хвост, а их хвосты состоят в отношении конкатенации с Y.

Для выполнения конкатенации над списками необходимо сформировать \emph{запрос} (или \emph{цель}).
В запросе в аргументах указываются либо замкнутые термы, либо термы со свободными переменными. Результатом
выполнения является список подстановок для свободных переменных, при которых отношение выполняется
(когда свободных переменных нет, подстановка, соответственно, пустая).

На рисунке~\ref{fig:appendoExample} приведёт пример запроса, в котором мы хотим найти возможные значения
переменных \lstinline{Y} и \lstinline{R}. Потенциально может быть бесконечное число ответов, к примеру,
когда все аргументы в запросе --- переменные, поэтому в системах miniKanren есть возможность
запрашивать несколько первых ответов; в примере, это число 1. Ответы могут содержать в себе как
конкретные замкнутые термы (к примеру, числа), так и свободные переменные, которые
в примере обозначаются как $\text{\_.}_n$. В примере одна и также свободная переменная $\text{\_.}_0$
назначена и \lstinline{Y} и \lstinline{R}. Это означает, что какое бы ни было значение \lstinline{Y}, оно всегда
будет являться хвостом \lstinline{R}.

\begin{figure}[h!]
\begin{lstlisting}[mathescape,language=Haskell,extendedchars=\true,frame=single,basicstyle=\ttfamily]
> run 1 (Y R) ($\text{append}^o$ [1, 2] Y R)
Y = $\text{\_.}_0$
R = 1 :: 2 :: $\text{\_.}_0$
\end{lstlisting}
\caption{Пример запуска отношения конкатенации.}
\label{fig:appendoExample}
\end{figure}

В определение miniKanren входит особый алгоритм поиска ответов --- чередующийся поиск \origin{interleaving search},
основанный на поиске в глубину, который рассматривает всё пространство поиска и гарантирует,
что если существует ответ, то алгоритм его предоставит за конечное время.
Для сравнения, обычный поиск в глубину, используемый в классическом Prolog при методе резолюций, может зациклиться
перед тем, как предоставить все ответы. Это свойство чередующегося поиска определяет, вместе
с отсутствием нечистых расширений, реляционность miniKanren. 

% Применимость
Хотя miniKanren уже применяется в индустрии для поиска лечения редких генетических
заболеваний в точной медицине\cite{medMK},
на данном этапе своего развития используется в основном в исследовательских целях.

Одно из интересных применений miniKanren --- {\it реляционные интерпретаторы}.

Для языка $L$ его интерпретатор -- это функция $\text{eval}_L$, которая принимает
на вход программу $p_L$ на этом языке, её вход $i$ и возвращает некоторый выход $o$
\[ \text{eval}_L (p_L, i) \equiv \llbracket p_L \rrbracket (i) = o,\]

где $\llbracket \bullet \rrbracket $ --- семантика языка L.
Тогда в miniKanren интерпретатор описывается отношением $\text{eval}^o_L(p_L, i, o)$.
При запуске такого отношения на разном наборе аргументов можно добиться интересных эффектов:
\begin{itemize}
\item по программе $p_L$ и выходу $o$ искать возможные входы $i$ (запуск программы в обратном направлении);
  %: $\text{eval}^o(\text{add}, I, 3)$
\item решать задачи поиска по задаче распознавания~\cite{lozov};
\item генерировать программы по заданной спецификации входа $i$ и выхода $o$
(техника программирования по примерам)\cite{unifiedMK}.
\end{itemize}

\todo{проблемы miniKanren: долгое вычисления сложных алгоритмов, долгие вычисления в обратном направлении.}

Одно из возможных решения проблем производительности --- специализация.


\subsection{Специализация}

{\bf Специализация программ} --- это метод автоматической оптимизации программ,
при которой из программы удаляются избыточные вычисления, зависимые от частично известного входа.
Специализацию программ также называют \emph{частичными} или \emph{смешанными вычислениями}\cite{jones}.

{\it Специализатор} $\text{spec}_L$ языка $L$ принимает на вход программу $p_L$ и часть известного входа этой
программы $i_s$ (\emph{статических} данных) и генерирует новую программу $\hat{p}_L$, которая ведёт себя на оставшемся
входе $i_d$ (\emph{динамических} данных) также, как и оригинальная программа (формула~\ref{eq:spec}).

\begin{equation}
  \llbracket \text{spec}_L(p_L, i_s) \rrbracket (i_d) \equiv \hat{p}_L (i_d) \equiv \llbracket p_L \rrbracket (i_s, i_d)
\label{eq:spec}
\end{equation}

% Эффекты специализаторов
Специализатор производит все вычисления, зависимые от статических данных,
протягивание констант, инлайнгинг и другие.


Одно из интересных теоретических применений специализации --- это \emph{проекции Футамуры}\cite{futamura}.
Процесс специализации интерпретатора на программу на языке $L$ $\text{spec}_L(\text{eval}_L, p_L)$
порождает \emph{скомплированную} программу $\hat{p}_L$, а процесс специализации специализатора
на интерпретатор языка $L$ $\text{spec}_{L''}(\text{spec}_{L'}, \text{eval}_L)$, в свою очередь,
порождает \emph{компилятор}. Это первая и вторая проекции Футамуры соответственно.
Однако реализация специализаторов, которые бы не оставляли в порождаемой программе
следы интерпретации, сложная и труднодостижимая задача\cite{jones}.

Специализация разделяется на два больших класса: \emph{online} и \emph{offline} алгоритмы:
\begin{itemize}
\item offline-cпециализаторы --- это двухфазовые алгоритмы специализации, в первой фазе
которого происходит разметка исхного кода, к примеру, с помощью анализа времени связывания\cite{jones},
и во второй фазе --- непосредственно во время специализации --- на основе полученной
разметки принимаются решения об оптимизации;
\item online-специализаторы, напротив, принимают решения о специализации на лету.
\end{itemize}

\todo{про то, какие специализаторы когда выгоднее применять (Jones, 147 page)}
\todo{Примеры}

{\bf Частичная дедукция} --- класс методов специализации логический языков,
основанное на построении деревьев вывода, отражающих процесс вывода методом резолюций,
и анализе отдельно взятых атомов логических формул\cite{advanced}.
\Cpd --- одно из расширений метода частичной дедукции, отличительная особенность которой
состоит в том, что конъюнкции рассматриваются как единая сущность наравне с атомами\cite{cpd}.
С помощью \forcpd возможно добиться различных оптимизационных эффектов, среди которых
выделяется дефорестация и таплинг.

Реализация методов частичной дедукции, включая конъюнктивную частичную дедукцию, для Prolog
представлена в виде системы ECCE\cite{ecce}.

В работе~\cite{lozov} представляется адаптация конъюнктивной частичной дедукции для miniKanren.
Реализация добивается существенного роста производительности, однако,
как будет показано в разделе~\ref{src:testing}, в силу особенностей метода и его
направленности на Prolog, нестабильно даёт хорошие результаты и
в некоторых случаях может затормозить исполнение программы.


\subsection{Суперкомпиляция}

{\bf Суперкомпиляция} --- метод анализа и преобразования программ,
который отслеживает обобщённую возможную историю вычислений исходной программы и строит
на её основе эквивалентную ему программу, структура которой, в некотором смысле, ``проще''
структуры исходной программы\cite{turchinSC}. % Турчин

Это упрощение достигаются путём удаления или преобразования
некоторых избыточных действий: удаление лишнего кода, выполнение операций над
уже известными данными, избавление от промежуточных структур данных, инлайнинг и т.п.

Суперкомпиляция включает в себя частичные вычисления, однако не сводится к ним полностью
и может привести в глубоким структурным изменениям оригинальной программы.

Суперкомпиляторы, которые используют только ``положительную'' информацию
--- то есть информацию о том, что сводобные переменные чему-то равны, ---
называют позитивными~\origin{positive supercompilation}\cite{scPos}.
К примеру, при достижении условного выражения {\bf if} x $=$ a {\bf then} $t_1$ {\bf else} $t_2$
позитивный суперкомпилятор при вычислении $t_1$ будет учитывать то, что x $=$ a,
однако при вычислении $t_2$ он не будет знать, что x $\neq$ a.
Расширение позитивного компилятора c поддержкой такой ``негативной'' информации --- идеальный
суперкомпилятор~\origin{perfect supercompilation}\cite{scPerf}.

% Про "символьное исполнение" https://en.wikipedia.org/wiki/Symbolic_execution
% Сделать ссылку на Ключникова?
%История вычислений представляется в виде {\it графа конфигураций}, где {\it конфигурация}
%описывает состояние вычисления на конкретном шаге. Построение графа происходит на этапе
%{\it прогонки}, во время которого происходит символьное исполнение программы. Потенциально
%такой граф --- который без дополнительных шагов прогонки вырождается в дерево ---
%бесконечный. Для трансформации бесконечного дерева в конечный объект используется {\it cвёртка}
%--- при обработке конфигурации, выражение в которой является {\it переименованием} выражения в одной
%из родительских конфигураций.

История вычислений представляется в виде \emph{графа процессов} --- корневого ориентированного графа,
в котором каждая ветвь --- это отдельный путь вычислений, а каждый узел --- состояние системы или \emph{конфигурация}.
Конфигурация обобщённо описывает множество состояний вычислительной системы или её подсистемы.
% К примеру, конфигурацией можно назвать пару из выражения $1 + x$, которое описывает все возможные суммы с $1$
% свободной переменной $x$, и множество органичений $\{ x \neq 10, x = 1 + x_1 \}$,
% которое сужает множество описываемых состояний до необходимого.
К примеру, конфигурацией можно назвать выражение $1 + x$, в котором параметр $x$ пробегает
все возможные значения своего домена (допустим, множество натуральных чисел) и задаёт
таким образом множество состояний программы. 

Процесс построение графа процессов называется \emph{прогонкой}~\origin{driving}.
Во время прогонки производится шаг символьных вычислений, после которого
в граф процессов добавляются порождённые конфигурации; множество конфигураций
появляются тогда, когда ветвления в программе зависят от свободных переменных.

В процессе прогонки в конфигурациях могут появляться новые свободные переменные,
которые строятся из исходной конфигурации:
если при вычислении выражение его переменная перешла в другую переменную (к примеру, из-за сопоставления с образцом),
то в итоговую конфигурацию будет подставлена новая переменная и связь старой и новой сохранится в
некоторой \emph{подстановке}.
Подстановка --- это отображение из множества переменных в множество возможно замкнутых термов.
Применение подстановки к выражению заменит все вхождения переменных, принадлежащих её домену,
на соответствующие термы. \todo{Что-нибудь ещё}

Пример графа процессов представлен на рисунке~\ref{fig:pgraphExample}.

\begin{figure}[h!]
\center
\begin{tikzpicture}[->,node distance=3cm, sibling distance=5cm]
                                                            
  \tikzstyle{conf}=[rectangle,draw, rounded corners=.8ex]

  \node[conf] (root) {$(a + b) + c$} ;
  \node[conf] (childLeft) [below left of = root] {$b + c$};
  \node[conf] (childRight)[below right of = root] {$(\text{Succ}(a_1) + b) + c$};
  \path (root) edge node[above left,pos=1] {$\{a \mapsto \text{Zero}\}$} (childLeft)
        (root) edge node[above right,pos=1]{$\{a \mapsto \text{Succ}(a_1)\}$}(childRight);
\end{tikzpicture}

\label{fig:pgraphExample}
\caption{Пример части графа процессов.}
\end{figure}

Потенциально процесс прогонки бесконечный, к примеру, когда происходят рекурсивные вызовы.
Для превращения бесконечого дерева вычисления в конечный объект, по которому множно
восстановить исходное дерево, используется \emph{свёртка.}

Свёртка~\origin{folding}~--- это процесс преобразования дерева процессов в граф, при котором
из вершины $v_c$ добавляется ребро в родительскую вершину $v_p$,
если выражение в конфигурации в $v_c$ и в $v_p$ равны с точностью до переименования.
Пример ситуации для свёртки изображён на рисунке~\ref{fig:pgraphFoldingExample},
на котором свёрточное ребро изображено пунтктиром.

\begin{figure}[h!]
\center
\begin{tikzpicture}[->,node distance=2.3cm, sibling distance=5cm]
                                                            
  \tikzstyle{conf}=[rectangle,draw, rounded corners=.8ex]

  \node[conf] (root) {$(a + b) + c$} ;
  \node[conf] (childLeft) [below left of = root] {$b + c$};
  \node[conf] (childRight)[below right of = root] {$(\text{Succ}(a_1) + b) + c$};
  \node[conf] (childRight2)[below  of = childRight] {$\text{Succ}((a_1 + b) + c)$};
  \node[conf] (childRight3)[below  of = childRight2] {$(a_1 + b) + c$};
  \node (left)[below of = childLeft] {$\cdots$};

  \path (root) edge node[above left,pos=1] {$\{a \mapsto \text{Zero}\}$} (childLeft)
        (root) edge node[above right,pos=1]{$\{a \mapsto \text{Succ}(a_1)\}$}(childRight)
        (childLeft) edge (left)
        (childRight) edge (childRight2)
        (childRight2) edge (childRight3)
        (childRight3) edge[bend right=90] (root);
\end{tikzpicture}

\label{fig:pgraphFoldingExample}
\caption{Пример свёртки.}
\end{figure}

Однако существует ситуации, при котором свёртка не привёт к тому, что граф превратится в
конечный объект. Такое может произойти, к примеру, когда два выражения структурно
схожи, но не существует переименования, уравнивающих их: $a + b$ и $a + a$.

Для решения этой проблемы используется \emph{обобщение}\cite{scGen}. Обобщение --- это процесс
замены одной конфигурации на другую, более абстрактную, описывающую больше состояний
программы. Для обнаружения ``похожей'' конфигурации используется предикат,
традиционно называемый \emph{свистком}. Сам шаг обобщения может произвести дейтсвия трёх видов:
\begin{itemize}
\item \emph{обобщение вниз} приводит к тому, что новая конфигурация заменяет текущую в графе процессов;
\item \emph{обобщение вверх} приводит к замене родительской конфигурации на обобщённую;
\item \emph{разделение}~\origin{split} используется для декомпозиции выражений, которые затем
будут обработаны отдельно.
\end{itemize}

\todo{вроде надо бы что-нибудь ещё написать про это вот всё}


%\newpage
\section{Специализация miniKanren}

В данной работе для специализации был выбран \ukanren --- минималистичный диалект языка miniKanren\cite{uKanren}.
\todo{todo}

Абстрактный cинтаксис языка представлен на Рисунке~\ref{fig:syntax}.

\begin{figure}[h!]
\centering
\[\begin{array}{ccll}
  \mathcal{C}   & = & \{C_i\}                                                   &\mbox{конструктор с арностью}\ i \\
  \mathcal{X}   & = & \{ x, y, z, \dots \}                                      &\mbox{переменные} \\
  \mathcal{T}_X & = & X \cup \{C_i (t_1, \dots, t_i) \mid t_j\in\mathcal{T}_X\} &\mbox{термы над множеством переменных $X$} \\
  \mathcal{D}   & = & \mathcal{T}_\emptyset                                     &\mbox{замкнутое выражение}\\
  \mathcal{R}   & = & \{ R_i\}                                                  &\mbox{реляционный символ с арностью}\ i \\[2mm]
  \mathcal{G}   & = & \mathcal{T_X}\equiv\mathcal{T_X}                          &\mbox{унификация} \\
                &   & \mathcal{G}\land\mathcal{G}                               &\mbox{конъюнкция} \\
                &   & \mathcal{G}\lor\mathcal{G}                                &\mbox{дизъюнкция} \\
                &   & \mbox{\lstinline|fresh|}\;\mathcal{X}\;.\;\mathcal{G}     &\mbox{введение свежей переменной} \\
                &   & R_i (t_1,\dots,t_i),\;t_j\in\mathcal{T_X}                 &\mbox{вызов реляционного отношения} \\[2mm]
  \mathcal{S}   & = & \{R_i^j = \lambda\;x_1\dots x_i\,.\, g_j;\}\; g           &\mbox{спецификация программы}
\end{array}\]
\caption{Синтаксис языка \ukanren.}
\label{fig:syntax}
\end{figure}

\begin{itemize}
\item Унификация двух термов $t_1 \equiv t_2$ порождает подстановку $\theta$, называемую \emph{унификатором},
      такую что её применение к термам уравнивает их: $t_1 \theta = t_2 \theta$.

      Алгоритм унификации находит наиболее общий унификатор \origin{most general unifier, mgu}, то есть такой
      унификатор $\theta$, что для любого другого унификатора $\theta'$ существует подстановку $\sigma$,
      с которой композиция наиболее общего унификатора даёт $\theta'$: $\theta' = \sigma \circ \theta$\cite{unification}.
      \todo{пояснить зачем здесь этот абзац}

      Алгоритм унификации языков семейства miniKanren использует проверку вхождения \origin{occurs check},
      что гарантирует корректность получаемых унификаторов, однако довольно сильно замедляет выполнение программ.

\item Конъюнкция двух целей $g_1 \land g_2$ подразумевает одновременное успешное выполнение выражений $g_1$ и $g_2$.
\item Дизъюнкция двух целей $g_1 \lor g_2$ подразумевает, что достаточно, чтобы хотя бы одно из выражений $g_1$ или $g_2$ выполнялось успешно.
      Следует отметить, что при выполнении $g_1$ выражение $g_2$ также будет вычисляться.
\item Введение свежей переменной $\text{fresh}\ x\ .\ g$ в языках miniKanren нужно указывать явно, в отличие, к примеру,
      от Prolog, где это происходит неявно.
\item Вызов реляционного отношения приводит к тому, что переданные в отношение термы
      унифицируются со аргументами отношения и подставляются в тело отношения. 
\end{itemize}


\subsection{Реализация суперкомпиляции}
Реализация суперкомпилятора для \ukanren строилась на основе проекта
по специализации miniKanren на основе конъюнктивной частичной
дедукции\footnote{\url{https://github.com/kajigor/uKanren_transformations/}}.


%\section{Алгоритм суперкомпиляции}
% \begin{figure}[h!]
% \begin{lstlisting}[mathescape,language=MyLang]
%   data Tree where
%     Failure     :: Tree
%     Success     :: Substitution $\rarrow$ Tree
%     Abstraction :: Expression $\rarrow$ Substitution $\rarrow$ Tree
%     Unfolding   :: Expression $\rarrow$ Substitution $\rarrow$ Tree
%     Renaming    :: Expression $\rarrow$ Substitution $\rarrow$ Tree
% \end{lstlisting}
% \end{figure}
% 
% 
% \begin{figure}[h!]
% \begin{algorithmic}
% \Function{drive}{tree, expression}
%   \If {expression is renaming of some parent in the tree}
%     \State {add leaf to tree}
%   \ElsIf {$\exists \text{parent}:$ parent is embeded in expression}
%     \State {add abstraction node in tree}
%     \State {abstracted-expr <- abstract expression (parents in tree)}
%     \ForAll{expr : abstracted-expr}
%       \State \Call{drive}{tree, expr}
%     \EndFor
%   \Else
%     \State {children <- unfold expression}
%     \State {add unfolding node in tree}
%     \ForAll{child : children}
%       \State \Call{drive}{tree, child}
%     \EndFor
%   \EndIf
% \EndFunction
% \end{algorithmic}
% \caption{Базовый алгоритм прогонки}
% \end{figure}
% 

%\newpage
\input{sec_results.tex}
%\newpage
%\input{src/sec_concl.tex}

\end{document}
