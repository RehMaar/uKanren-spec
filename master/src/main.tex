\documentclass[14pt, a4paper] {extarticle}
\usepackage[utf8] {inputenc}
\usepackage[T2A]{fontenc}
\usepackage[english, russian] {babel}
\usepackage[top=20mm, bottom=20mm, left=30mm, right=10mm]{geometry}


%\usepackage{hyperref}
\usepackage{cmap}

% Чтобы зачёркивать символы
\usepackage[makeroom]{cancel}

\usepackage{xcolor}
\usepackage{tikz}

\usepackage{biblatex}

% Для вставки фигур
\usepackage{float}
\usepackage{caption}


\usepackage{longtable,amsmath,amsfonts,amssymb}

\usepackage{listings}
\usepackage{algpseudocode}
% More Math Symbols
\usepackage{stmaryrd}

% Enumeration
\usepackage{enumitem}
\setlist[enumerate]{topsep=0pt,itemsep=0ex,partopsep=1ex,parsep=1ex}
\setlist[itemize]{itemsep=0ex}

% Полуторный межстрочный интервал
\usepackage[nodisplayskipstretch]{setspace}
\onehalfspacing
%\renewcommand{\baselinestretch}{1.5}
%\linespread{1.3}

% Добавить абзацный отступ для первых абзацев в section/subsection,
% по умолчанию не добавляется
\usepackage{indentfirst}

% Красивый маркер ненумерованного списка в виде тире
\def\labelitemi{--}

% Times New Roman
\usepackage{pscyr}
\renewcommand{\rmdefault}{ftm}


% Абзацный отступ равен 1.25 см
\parindent=1.25cm

\addto\captionsrussian{
% "Оглавление" вместо "Содержание"
% подпись "Рисунок" вместо "Рис"
  \def\figurename{{Рисунок}}
}

% Каждый пункт оглавления должен быть с отточием
\usepackage{titletoc}


% Максимальная вложенность содержания (только разделы, подразделы и
% "пункты")
\setcounter{tocdepth}{3}

% Возможность переопределять оглавление и его стиль
\usepackage{tocloft}
% Слово "Оглавление" заглавными буквами
\makeatletter
\patchcmd{\@cftmaketoctitle}{\cfttoctitlefont\contentsname}{\cfttoctitlefont\MakeUppercase{\contentsname}}{}{}
\makeatother

% Номер страницы по середине верхнего поля
\usepackage{fancyhdr}
\pagestyle{fancy}
\fancyhf{}
\fancyhead[C]{\thepage}
\renewcommand{\headrulewidth}{0pt}



% Самое длинное тире в качестве разделителя в подписях к рисункам,
% таблицам, листингам и др.
\DeclareCaptionLabelSeparator{emdash}{ --- }
\captionsetup{labelsep=emdash}

% Подпись к таблице должна быть по левому краю
\captionsetup[table]{singlelinecheck=false}

% Сквозная нумерация таблиц, формул, рисунков
%\renewcommand{\theequation}{\arabic{equation}}
%\renewcommand{\thetable}{\arabic{table}}
%\renewcommand{\thefigure}{\arabic{figure}}



% Обязательно переносить слова, чтобы соблюсти поля документа. Для
% соблюдения полей можно пренебречь правилами для тех слов и
% словосочетаний, о которых не знают словаря переносов (ruhyphen или
% ruenhyph). Оно почему-то работает. Взято с:
%
%   http://www.latex-community.org/forum/viewtopic.php?p=70342#p70342
%
\tolerance 1414
\hbadness 1414
\emergencystretch 1.5em
\hfuzz 0.3pt
\widowpenalty=10000
\vfuzz \hfuzz
\raggedbottom

%%%%%%%%%%%%%%%%%%%%%%%%%
\lstdefinelanguage{MyLang}
{
  morekeywords={data},
  keywordstyle=\bfseries\color{black}
}

%\lstset{language=Haskell}
\lstset {
mathescape,
extendedchars=\true,
basicstyle=\ttfamily,
numberstyle=\ttfamily,
numbers=left,
stepnumber=1,
frame=single,
morekeywords={data, do, type, if, then, else, class, where, return}
}

%%%%%%%%%%%%%%%%%%%%%%%%%
\newcommand{\larrow}{\leftarrow}
\newcommand{\rarrow}{\rightarrow}

\newcommand{\embed}{\unlhd}
\newcommand{\inst}{\preccurlyeq}
\newcommand{\strictinst}{\prec}
% \newcommand{\variant}{\leftrightarrow}
\newcommand{\variant}{\approx}
\newcommand{\genup}{\lessdot}

\newcommand{\ukanren}{$\mu$Kanren }
\newcommand{\forcpd}{конъюнктивной частичной дедукции }
\newcommand{\cpd}{конъюнктивная частичная дедукция }
\newcommand{\Cpd}{Конъюнктивная частичная дедукция }

\newcommand{\origin}[1]{(англ. {\it #1})}

\newcommand{\rel}[1]{$\text{#1}^o$}

\newcommand{\todo}[1]{{\bf\color{red}TODO: #1}}
%%%%%%%%%%%%%%%%%%%%%%%%%

\addbibresource{main.bib}

\begin{document}
\setcounter{figure}{0}
%\setcounter{page}{3}  %% Why??????
\tableofcontents
\newpage
\phantomsection
\section*{ВВЕДЕНИЕ}
\addcontentsline{toc}{section}{ВВЕДЕНИЕ}

% ВВЕДЕНИЕ + ЗАКЛЮЧЕНИЕ -- описание такое, чтобы было понятно о чём работа и
% какие результаты
% В общих чертах вся работа, указать про аналоги

Реляционное программирование~--- это чистая форма логического программирования,
в которой программы представляются как наборы математических отношений~\cite{byrdMK}.
Отношения
не делают различие между выходными и выходными параметрами, из-за чего одно и то же
отношение может покрыть несколько связанных проблем. К примеру, отношение, задающее
интерпретатор языка, можно использовать не только для вычисления программ по
заданному входу, но и генерировать возможные входные значения по заданному результату
или генерировать сами программы по спецификации входных и выходных значений интерпратетора.

miniKanren~--- это семейство встраиваемых предметно-ориентированных языков программирования~\cite{byrdMK}.
miniKanren был специально сконструирован для того, чтобы поддержать реляционную парадигму,
опираясь на опыт логических языков, таких как языков семейства Prolog~\cite{logicMJ},
Mercury~\cite{mercury} или Curry~\cite{curry}.

Однако реляционная парадигма довольно сложна, хотя потенциал её весьма велик.
Часто наиболее простой способ записи отношения не является эффективным. В
частности, при задании функциональных отношений как сопоставления выходов
входам, как это наблюдается в примере с интерпратетором, практически
всегда работает медленно.

Специализация -- это техника автоматической оптимизации программ,
при которой на основе программы и её возможно частично известного входа
порождается новая, более оптимальная программа, которая сохраняет семантику
исходной. Для специализации логических языков используются методы частичной дедукции~\cite{advanced},
самый проработанный из которых --- это \cpd\cite{cpd}. Реализация \cpd для Prolog ECCE показывает
хорошие результаты~\cite{controlPoly}, однако специфика реляционного программирования
и отличие его от логических языков подразумевает возможность разработать более подходящий
метод специализации. Уже существует адаптация \cpd для miniKanren~\cite{lozov},
однако её результаты нестабильны, и несмотря на то, что в некоторых случаях
производительность программ улучшается, в других -- она может существенно ухудшиться.

Другой подход для специализации --- это суперкомпиляция, 
техника автоматической трансформации и анализа программ,
при которой программа символьно исполняется с сохранением истории вычислений,
на основе которой принимаются решения о трансформациях.
Суперкомпиляция успешно применяется к функциональным и императивным языкам,
однако суперкомпиляция для логическим языкам не сильно развита, хотя и существуют
работы, посвящённые демонстрации сходства процессов частичной дедукции и суперкомпиляции~\cite{pdAndDriving},
а также суперкомпилятор APROPOS~\cite{apropos}, который, однако, довольно ограничен
в своих возможностях и требует ручного контроля.

В данной работе предлагается способ адаптации и реализации суперкомпилятора для
реляционного языка miniKanren, также рассматриваются его возможные вариации для
повышения производительности реляционных программ и производится апробация результата.

\newpage
\section{Описание предметной области}


\subsection{Существующие решения}


\subsection{Постановка задачи}



\newpage
\section{Имеющиеся наработки}

В этом разделе описывается более подробно метод суперкомпиляции,
а также библиотека для специализации miniKanren, на основе
которой велась разработка.

\subsection{Алгоритмы cуперкомпиляции}

{\bf Суперкомпиляция} --- метод анализа и преобразования программ,
который отслеживает обобщённую возможную историю вычислений исходной программы и строит
на её основе эквивалентную ему программу, структура которой, в некотором смысле, ``проще''
структуры исходной программы\cite{turchinSC}. % Турчин

Это упрощение достигаются путём удаления или преобразования
некоторых избыточных действий: удаление лишнего кода, выполнение операций над
уже известными данными, избавление от промежуточных структур данных, инлайнинг и т.п.

Суперкомпиляция включает в себя частичные вычисления, однако не сводится к ним полностью
и может привести в глубоким структурным изменениям оригинальной программы.

Суперкомпиляторы, которые используют только ``положительную'' информацию
--- то есть информацию о том, что сводобные переменные чему-то равны, ---
называют позитивными~\origin{positive supercompilation}\cite{scPos}.
К примеру, при достижении условного выражения {\bf if} x $=$ a {\bf then} $t_1$ {\bf else} $t_2$
позитивный суперкомпилятор при вычислении $t_1$ будет учитывать то, что x $=$ a,
однако при вычислении $t_2$ он не будет знать, что x $\neq$ a.
Расширение позитивного компилятора c поддержкой такой ``негативной'' информации --- идеальный
суперкомпилятор~\origin{perfect supercompilation}\cite{scPerf}.

% Про "символьное исполнение" https://en.wikipedia.org/wiki/Symbolic_execution
% Сделать ссылку на Ключникова?
%История вычислений представляется в виде {\it графа конфигураций}, где {\it конфигурация}
%описывает состояние вычисления на конкретном шаге. Построение графа происходит на этапе
%{\it прогонки}, во время которого происходит символьное исполнение программы. Потенциально
%такой граф --- который без дополнительных шагов прогонки вырождается в дерево ---
%бесконечный. Для трансформации бесконечного дерева в конечный объект используется {\it cвёртка}
%--- при обработке конфигурации, выражение в которой является {\it переименованием} выражения в одной
%из родительских конфигураций.

История вычислений представляется в виде \emph{графа процессов} --- корневого ориентированного графа,
в котором каждая ветвь --- это отдельный путь вычислений, а каждый узел --- состояние системы или \emph{конфигурация}.
Конфигурация обобщённо описывает множество состояний вычислительной системы или её подсистемы.
% К примеру, конфигурацией можно назвать пару из выражения $1 + x$, которое описывает все возможные суммы с $1$
% свободной переменной $x$, и множество органичений $\{ x \neq 10, x = 1 + x_1 \}$,
% которое сужает множество описываемых состояний до необходимого.
К примеру, конфигурацией можно назвать выражение $1 + x$, в котором параметр $x$ пробегает
все возможные значения своего домена (допустим, множество натуральных чисел) и задаёт
таким образом множество состояний программы. 

Процесс построение графа процессов называется \emph{прогонкой}~\origin{driving}.
Во время прогонки производится шаг символьных вычислений, после которого
в граф процессов добавляются порождённые конфигурации; множество конфигураций
появляются тогда, когда ветвления в программе зависят от свободных переменных.

В процессе прогонки в конфигурациях могут появляться новые свободные переменные,
которые строятся из исходной конфигурации:
если при вычислении выражение его переменная перешла в другую переменную (к примеру, из-за сопоставления с образцом),
то в итоговую конфигурацию будет подставлена новая переменная и связь старой и новой сохранится в
некоторой \emph{подстановке}.
Подстановка --- это отображение из множества переменных в множество возможно замкнутых термов.
Применение подстановки к выражению заменит все вхождения переменных, принадлежащих её домену,
на соответствующие термы. \todo{Что-нибудь ещё}

Пример графа процессов представлен на рисунке~\ref{fig:pgraphExample}.

\begin{figure}[h!]
\center
\begin{tikzpicture}[->,node distance=3cm, sibling distance=5cm]
                                                            
  \tikzstyle{conf}=[rectangle,draw, rounded corners=.8ex]

  \node[conf] (root) {$(a + b) + c$} ;
  \node[conf] (childLeft) [below left of = root] {$b + c$};
  \node[conf] (childRight)[below right of = root] {$(\text{Succ}(a_1) + b) + c$};
  \path (root) edge node[above left,pos=1] {$\{a \mapsto \text{Zero}\}$} (childLeft)
        (root) edge node[above right,pos=1]{$\{a \mapsto \text{Succ}(a_1)\}$}(childRight);
\end{tikzpicture}

\label{fig:pgraphExample}
\caption{Пример части графа процессов.}
\end{figure}

Потенциально процесс прогонки бесконечный, к примеру, когда происходят рекурсивные вызовы.
Для превращения бесконечого дерева вычисления в конечный объект, по которому множно
восстановить исходное дерево, используется \emph{свёртка.}

Свёртка~\origin{folding}~--- это процесс преобразования дерева процессов в граф, при котором
из вершины $v_c$ добавляется ребро в родительскую вершину $v_p$,
если выражение в конфигурации в $v_c$ и в $v_p$ равны с точностью до переименования.
Пример ситуации для свёртки изображён на рисунке~\ref{fig:pgraphFoldingExample},
на котором свёрточное ребро изображено пунтктиром.

\begin{figure}[h!]
\center
\begin{tikzpicture}[->,node distance=2.3cm, sibling distance=5cm]
                                                            
  \tikzstyle{conf}=[rectangle,draw, rounded corners=.8ex]

  \node[conf] (root) {$(a + b) + c$} ;
  \node[conf] (childLeft) [below left of = root] {$b + c$};
  \node[conf] (childRight)[below right of = root] {$(\text{Succ}(a_1) + b) + c$};
  \node[conf] (childRight2)[below  of = childRight] {$\text{Succ}((a_1 + b) + c)$};
  \node[conf] (childRight3)[below  of = childRight2] {$(a_1 + b) + c$};
  \node (left)[below of = childLeft] {$\cdots$};

  \path (root) edge node[above left,pos=1] {$\{a \mapsto \text{Zero}\}$} (childLeft)
        (root) edge node[above right,pos=1]{$\{a \mapsto \text{Succ}(a_1)\}$}(childRight)
        (childLeft) edge (left)
        (childRight) edge (childRight2)
        (childRight2) edge (childRight3)
        (childRight3) edge[bend right=90] (root);
\end{tikzpicture}

\label{fig:pgraphFoldingExample}
\caption{Пример свёртки.}
\end{figure}

Однако существует ситуации, при котором свёртка не привёт к тому, что граф превратится в
конечный объект. Такое может произойти, к примеру, когда два выражения структурно
схожи, но не существует переименования, уравнивающих их: $a + b$ и $a + a$.

Для решения этой проблемы используется \emph{обобщение}\cite{scGen}. Обобщение --- это процесс
замены одной конфигурации на другую, более абстрактную, описывающую больше состояний
программы. Для обнаружения ``похожей'' конфигурации используется предикат,
традиционно называемый \emph{свистком}. Сам шаг обобщения может произвести дейтсвия трёх видов:
\begin{itemize}
\item \emph{обобщение вниз} приводит к тому, что новая конфигурация заменяет текущую в графе процессов;
\item \emph{обобщение вверх} приводит к замене родительской конфигурации на обобщённую;
\item \emph{разделение}~\origin{split} используется для декомпозиции выражений, которые затем
будут обработаны отдельно.
\end{itemize}

\todo{вроде надо бы что-нибудь ещё написать про это вот всё}


\subsection{Язык \ukanren}

В данной работе для специализации был выбран \ukanren --- минималистичный диалект языка miniKanren\cite{uKanren}.
\todo{todo}

Абстрактный cинтаксис языка представлен на Рисунке~\ref{fig:syntax}.

\begin{figure}[h!]
\centering
\[\begin{array}{ccll}
  \mathcal{C}   & = & \{C_i\}                                                   &\mbox{конструктор с арностью}\ i \\
  \mathcal{X}   & = & \{ x, y, z, \dots \}                                      &\mbox{переменные} \\
  \mathcal{T}_X & = & X \cup \{C_i (t_1, \dots, t_i) \mid t_j\in\mathcal{T}_X\} &\mbox{термы над множеством переменных $X$} \\
  \mathcal{D}   & = & \mathcal{T}_\emptyset                                     &\mbox{замкнутое выражение}\\
  \mathcal{R}   & = & \{ R_i\}                                                  &\mbox{реляционный символ с арностью}\ i \\[2mm]
  \mathcal{G}   & = & \mathcal{T_X}\equiv\mathcal{T_X}                          &\mbox{унификация} \\
                &   & \mathcal{G}\land\mathcal{G}                               &\mbox{конъюнкция} \\
                &   & \mathcal{G}\lor\mathcal{G}                                &\mbox{дизъюнкция} \\
                &   & \mbox{\lstinline|fresh|}\;\mathcal{X}\;.\;\mathcal{G}     &\mbox{введение свежей переменной} \\
                &   & R_i (t_1,\dots,t_i),\;t_j\in\mathcal{T_X}                 &\mbox{вызов реляционного отношения} \\[2mm]
  \mathcal{S}   & = & \{R_i^j = \lambda\;x_1\dots x_i\,.\, g_j;\}\; g           &\mbox{спецификация программы}
\end{array}\]
\caption{Синтаксис языка \ukanren.}
\label{fig:syntax}
\end{figure}

\begin{itemize}
\item Унификация двух термов $t_1 \equiv t_2$ порождает подстановку $\theta$, называемую \emph{унификатором},
      такую что её применение к термам уравнивает их: $t_1 \theta = t_2 \theta$.

      Алгоритм унификации находит наиболее общий унификатор \origin{most general unifier, mgu}, то есть такой
      унификатор $\theta$, что для любого другого унификатора $\theta'$ существует подстановку $\sigma$,
      с которой композиция наиболее общего унификатора даёт $\theta'$: $\theta' = \sigma \circ \theta$\cite{unification}.
      \todo{пояснить зачем здесь этот абзац}

      Алгоритм унификации языков семейства miniKanren использует проверку вхождения \origin{occurs check},
      что гарантирует корректность получаемых унификаторов, однако довольно сильно замедляет выполнение программ.

\item Конъюнкция двух целей $g_1 \land g_2$ подразумевает одновременное успешное выполнение выражений $g_1$ и $g_2$.
\item Дизъюнкция двух целей $g_1 \lor g_2$ подразумевает, что достаточно, чтобы хотя бы одно из выражений $g_1$ или $g_2$ выполнялось успешно.
      Следует отметить, что при выполнении $g_1$ выражение $g_2$ также будет вычисляться.
\item Введение свежей переменной $\text{fresh}\ x\ .\ g$ в языках miniKanren нужно указывать явно, в отличие, к примеру,
      от Prolog, где это происходит неявно.
\item Вызов реляционного отношения приводит к тому, что переданные в отношение термы
      унифицируются со аргументами отношения и подставляются в тело отношения. 
\end{itemize}


\subsection{Библиотека для специализации miniKanren}

Реализация суперкомпилятора для \ukanren строилась на основе проекта по специализации miniKanren с помощью конъюнктивной частичной
дедукции\footnote{\url{https://github.com/kajigor/uKanren_transformations/}} на функциональном языке программирования Haskell.
\todo{...}

%%%%%%%%%%%%%%%%%%%%%%%%%%%%%%%%%%%%%%%%%%%%%%%%%%%%%%%%%
%%%%%%%%%%%%%%%%%%%%%%%%%%%%%%%%%%%%%%%%%%%%%%%%%%%%%%%%%
% Про инстансы и прочее
%%%%%%%%%%%%%%%%%%%%%%%%%%%%%%%%%%%%%%%%%%%%%%%%%%%%%%%%%
%%%%%%%%%%%%%%%%%%%%%%%%%%%%%%%%%%%%%%%%%%%%%%%%%%%%%%%%%
Терм $t_2$ является \emph{экземпляром} \origin{instance} терма $t_1$, если
существует подстановка $\theta$, такая что $t_1 \theta = t_2$.

$t_2$ является \emph{строгим} экземпляром $t_1$, если  $t_2$ является экземпляром $t_1$ и
$t_1$ не является экземпляром $t_2$.
\todo{Почему нам это важно}

%%%%%%%%%%%%%%%%%%%%%%%%%%%%%%%%%%%%%%%%%%%%%%%%%%%%%%%%%
%%%%%%%%%%%%%%%%%%%%%%%%%%%%%%%%%%%%%%%%%%%%%%%%%%%%%%%%%
% Свисток и гомеоморфное вложение
%%%%%%%%%%%%%%%%%%%%%%%%%%%%%%%%%%%%%%%%%%%%%%%%%%%%%%%%%
%%%%%%%%%%%%%%%%%%%%%%%%%%%%%%%%%%%%%%%%%%%%%%%%%%%%%%%%%

В качестве свистка используется отношение \emph{гомеоморфного вложения}.\cite{scGen}
Отношение гомеоморфного вложения $\unlhd$ определено индуктивно:
\begin{itemize}
\item переменные вложены в переменные: $x \embed y$;
\item терм $X$ вложен в конструктор с именем $C$, если он вложен в один из аргументов конструктора:
      $$X \embed C_n(Y_1, \dots, Y_n): \exists i, X \embed Y_i;$$
\item конструкторы с одинаковыми именами состоят в отношении вложения, если в этом отношении
      состоят их аргументы:
      $$C_n(X_1, \dots, X_n) \embed C_n(Y_1, \dots, Y_n): \forall i, X_i \embed Y_i.$$
\end{itemize}

К примеру, выражение $c(b) \embed c(f(b))$, но $f(c(b)) \cancel{\embed} c(f(b))$.

Отношение строгого гомеоморфного вложения $\embed^*$ вводит дополнительное
требование, чтобы терм $X$, состоящий в отношении с $Y$, не был \emph{строгим экземпляром} $Y$. 

Преимущество использования гомеоморфного вложения, в первую очередь, состоит в том,
что для этого отношения доказано, что на бесконечной последовательности выражений $e_0, e_1, \dots, e_n$
обязательно найдутся такие два индекса $i < j$, что $e_i \embed e_j$, вне зависимости
от того, каким образом последовательность выражений была получена~\cite{scPos}.




%%%%%%%%%%%%%%%%%%%%%%%%%%%%%%%%%%%%%%%%%%%%%%%%%%%%%%%%%
%%%%%%%%%%%%%%%%%%%%%%%%%%%%%%%%%%%%%%%%%%%%%%%%%%%%%%%%%
% Обобщение
%%%%%%%%%%%%%%%%%%%%%%%%%%%%%%%%%%%%%%%%%%%%%%%%%%%%%%%%%
%%%%%%%%%%%%%%%%%%%%%%%%%%%%%%%%%%%%%%%%%%%%%%%%%%%%%%%%%

\emph{Наиболее тесное обобщение} \origin{most specific generalization} \todo{вот такое вот оно}.



\subsection{Обобщённый алгоритм суперкомпиляции}

\input{env/scgen.tex}

\newpage
\section{Суперкомпиляция miniKanren}
% \label{sec:scmk}
% В данной работе для специализации был выбран \ukanren --- минималистичный диалект языка miniKanren\cite{uKanren}.
\todo{todo}

Абстрактный cинтаксис языка представлен на Рисунке~\ref{fig:syntax}.

\begin{figure}[h!]
\centering
\[\begin{array}{ccll}
  \mathcal{C}   & = & \{C_i\}                                                   &\mbox{конструктор с арностью}\ i \\
  \mathcal{X}   & = & \{ x, y, z, \dots \}                                      &\mbox{переменные} \\
  \mathcal{T}_X & = & X \cup \{C_i (t_1, \dots, t_i) \mid t_j\in\mathcal{T}_X\} &\mbox{термы над множеством переменных $X$} \\
  \mathcal{D}   & = & \mathcal{T}_\emptyset                                     &\mbox{замкнутое выражение}\\
  \mathcal{R}   & = & \{ R_i\}                                                  &\mbox{реляционный символ с арностью}\ i \\[2mm]
  \mathcal{G}   & = & \mathcal{T_X}\equiv\mathcal{T_X}                          &\mbox{унификация} \\
                &   & \mathcal{G}\land\mathcal{G}                               &\mbox{конъюнкция} \\
                &   & \mathcal{G}\lor\mathcal{G}                                &\mbox{дизъюнкция} \\
                &   & \mbox{\lstinline|fresh|}\;\mathcal{X}\;.\;\mathcal{G}     &\mbox{введение свежей переменной} \\
                &   & R_i (t_1,\dots,t_i),\;t_j\in\mathcal{T_X}                 &\mbox{вызов реляционного отношения} \\[2mm]
  \mathcal{S}   & = & \{R_i^j = \lambda\;x_1\dots x_i\,.\, g_j;\}\; g           &\mbox{спецификация программы}
\end{array}\]
\caption{Синтаксис языка \ukanren.}
\label{fig:syntax}
\end{figure}

\begin{itemize}
\item Унификация двух термов $t_1 \equiv t_2$ порождает подстановку $\theta$, называемую \emph{унификатором},
      такую что её применение к термам уравнивает их: $t_1 \theta = t_2 \theta$.

      Алгоритм унификации находит наиболее общий унификатор \origin{most general unifier, mgu}, то есть такой
      унификатор $\theta$, что для любого другого унификатора $\theta'$ существует подстановку $\sigma$,
      с которой композиция наиболее общего унификатора даёт $\theta'$: $\theta' = \sigma \circ \theta$\cite{unification}.
      \todo{пояснить зачем здесь этот абзац}

      Алгоритм унификации языков семейства miniKanren использует проверку вхождения \origin{occurs check},
      что гарантирует корректность получаемых унификаторов, однако довольно сильно замедляет выполнение программ.

\item Конъюнкция двух целей $g_1 \land g_2$ подразумевает одновременное успешное выполнение выражений $g_1$ и $g_2$.
\item Дизъюнкция двух целей $g_1 \lor g_2$ подразумевает, что достаточно, чтобы хотя бы одно из выражений $g_1$ или $g_2$ выполнялось успешно.
      Следует отметить, что при выполнении $g_1$ выражение $g_2$ также будет вычисляться.
\item Введение свежей переменной $\text{fresh}\ x\ .\ g$ в языках miniKanren нужно указывать явно, в отличие, к примеру,
      от Prolog, где это происходит неявно.
\item Вызов реляционного отношения приводит к тому, что переданные в отношение термы
      унифицируются со аргументами отношения и подставляются в тело отношения. 
\end{itemize}

% 
% \subsection{Библиотека специализации}
% 
% Реализация суперкомпилятора для \ukanren строилась на основе проекта по специализации miniKanren с помощью конъюнктивной частичной
дедукции\footnote{\url{https://github.com/kajigor/uKanren_transformations/}} на функциональном языке программирования Haskell.
\todo{...}

%%%%%%%%%%%%%%%%%%%%%%%%%%%%%%%%%%%%%%%%%%%%%%%%%%%%%%%%%
%%%%%%%%%%%%%%%%%%%%%%%%%%%%%%%%%%%%%%%%%%%%%%%%%%%%%%%%%
% Про инстансы и прочее
%%%%%%%%%%%%%%%%%%%%%%%%%%%%%%%%%%%%%%%%%%%%%%%%%%%%%%%%%
%%%%%%%%%%%%%%%%%%%%%%%%%%%%%%%%%%%%%%%%%%%%%%%%%%%%%%%%%
Терм $t_2$ является \emph{экземпляром} \origin{instance} терма $t_1$, если
существует подстановка $\theta$, такая что $t_1 \theta = t_2$.

$t_2$ является \emph{строгим} экземпляром $t_1$, если  $t_2$ является экземпляром $t_1$ и
$t_1$ не является экземпляром $t_2$.
\todo{Почему нам это важно}

%%%%%%%%%%%%%%%%%%%%%%%%%%%%%%%%%%%%%%%%%%%%%%%%%%%%%%%%%
%%%%%%%%%%%%%%%%%%%%%%%%%%%%%%%%%%%%%%%%%%%%%%%%%%%%%%%%%
% Свисток и гомеоморфное вложение
%%%%%%%%%%%%%%%%%%%%%%%%%%%%%%%%%%%%%%%%%%%%%%%%%%%%%%%%%
%%%%%%%%%%%%%%%%%%%%%%%%%%%%%%%%%%%%%%%%%%%%%%%%%%%%%%%%%

В качестве свистка используется отношение \emph{гомеоморфного вложения}.\cite{scGen}
Отношение гомеоморфного вложения $\unlhd$ определено индуктивно:
\begin{itemize}
\item переменные вложены в переменные: $x \embed y$;
\item терм $X$ вложен в конструктор с именем $C$, если он вложен в один из аргументов конструктора:
      $$X \embed C_n(Y_1, \dots, Y_n): \exists i, X \embed Y_i;$$
\item конструкторы с одинаковыми именами состоят в отношении вложения, если в этом отношении
      состоят их аргументы:
      $$C_n(X_1, \dots, X_n) \embed C_n(Y_1, \dots, Y_n): \forall i, X_i \embed Y_i.$$
\end{itemize}

К примеру, выражение $c(b) \embed c(f(b))$, но $f(c(b)) \cancel{\embed} c(f(b))$.

Отношение строгого гомеоморфного вложения $\embed^*$ вводит дополнительное
требование, чтобы терм $X$, состоящий в отношении с $Y$, не был \emph{строгим экземпляром} $Y$. 

Преимущество использования гомеоморфного вложения, в первую очередь, состоит в том,
что для этого отношения доказано, что на бесконечной последовательности выражений $e_0, e_1, \dots, e_n$
обязательно найдутся такие два индекса $i < j$, что $e_i \embed e_j$, вне зависимости
от того, каким образом последовательность выражений была получена~\cite{scPos}.




%%%%%%%%%%%%%%%%%%%%%%%%%%%%%%%%%%%%%%%%%%%%%%%%%%%%%%%%%
%%%%%%%%%%%%%%%%%%%%%%%%%%%%%%%%%%%%%%%%%%%%%%%%%%%%%%%%%
% Обобщение
%%%%%%%%%%%%%%%%%%%%%%%%%%%%%%%%%%%%%%%%%%%%%%%%%%%%%%%%%
%%%%%%%%%%%%%%%%%%%%%%%%%%%%%%%%%%%%%%%%%%%%%%%%%%%%%%%%%

\emph{Наиболее тесное обобщение} \origin{most specific generalization} \todo{вот такое вот оно}.


% 
% \subsection{Обобщённый алгоритм суперкомпиляции}
% 
% \input{sc/scgen.tex}

\subsection{Реализация суперкомпилятора}

%%%%%%%%%%%%%%%%%%%%%%%%%%%%%%%%%%%%%%%%%%%%%%%%%%%%%%%%%%%%%%%%%%%
%%%%%%%%%%%%%%%%%%%%%%%%%%%%%%%%%%%%%%%%%%%%%%%%%%%%%%%%%%%%%%%%%%%
% Описание "графа" процессов
%%%%%%%%%%%%%%%%%%%%%%%%%%%%%%%%%%%%%%%%%%%%%%%%%%%%%%%%%%%%%%%%%%%
%%%%%%%%%%%%%%%%%%%%%%%%%%%%%%%%%%%%%%%%%%%%%%%%%%%%%%%%%%%%%%%%%%%

Представление графа процессов в Haskell затруднено тем, что графовые
структуры данных обычно требуют ссылок на произвольные узлы,
что приводит к появлению перекрёсных ссылок. Прямая реализация этой
идеи сложна в разработке и поддержке и не является идиоматичной.
Использование \emph{IORef}\footnote{https://hackage.haskell.org/package/base-4.11.1.0/docs/Data-IORef.html},
хотя и предоставляет мутабельность, приводит к неоправданному усложнению кода всего проекта,
избавляя код от функциональной чистоты.

Заметим, что графовость этой структуре данных придают \emph{обратные рёбра}
(то есть рёбра от детей к родителям), которые появляются при свёртке, когда ребёнок
является переименованием родителя.
Тогда, если уметь сохранять или восстанавливать информацию об этой связи, то
достаточным будет представить граф в качестве \emph{дерева} процессов.
Древовидные структура однозначно отображается на процесс символьных вычислений,
а также с ними легко и идиоматично работать в Haskell.

Структура дерева процессов представлена на рисунке~\ref{fig:ptree}.

\begin{figure}[h!]
\begin{lstlisting}[mathescape,language=Haskell,extendedchars=\true,frame=single,basicstyle=\ttfamily]
type Conf = Conjunction (RelationCall FreeVar)

type Subs = Variable $\mapsto$ Term

data Tree where
  Failure     :: Tree
  Success     :: Subst $\rarrow$ Tree
  Renaming    :: Conf $\rarrow$ Subst $\rarrow$ Tree
  Abstraction :: Conf $\rarrow$ Subst $\rarrow$ List Tree $\rarrow$ Tree
  Generalizer :: Subst $\rarrow$ Tree $\rarrow$ Tree
  Unfolding   :: Conf $\rarrow$ Subst $\rarrow$ List Tree $\rarrow$ Tree
\end{lstlisting}
\caption{Описание дерева процессов.}
\label{fig:ptree}
\end{figure}

Конфигурация \lstinline{Conf} определена как выражение со свободными переменными.
В узле дерева процессов хранится конфигурация, приведённая к форме, содержащей только конъюнкцию вызовов
реляционного отношения. Это сделано из тех соображений, что, во-первых, дизъюнкция представляет
собой ветвление вычислений, посему, соответственно, представляется как ветвление в дереве процессов,
во-вторых, унификации производятся во время символьных вычислений и добавляются в подстановку,
в-третьих, так как введение свежей переменной оказывает влияние лишь на состояние, в котором производятся вычисления,
неосмысленно сохранять его в конфигурации.

Подстановка \lstinline{Subst} соответствует своему математическому определению как отображение из
переменных в термы.
Узлы дерева процессов представляют шаги суперкомпиляции и исходы вычисления выражений:
\begin{itemize}
\item \lstinline{Failure} обозначает неудавшееся вычисления. Такой исход
      случается при появлении противоречивых подстановок;
\item \lstinline{Success}, напротив, обозначает удавшееся вычисление, которое свелось к подстановке \lstinline{Subst};
\item \lstinline{Renaming} обозначает узел, конфигурация которой является переименованием какого-то родительского узла.
\item \lstinline{Abstraction} обознает узел, который может быть обобщён на одного из родителей;
      После обобщения может появится несколько конфигураций, которые являются результатом применения разделения.
      Эти конфигурации добавляются в качестве списка дочерних поддеревьев в текущий узел;
\item \lstinline{Generalizer} хранит себе унификатор, который порождается во время обобщения
      двух термов, и поддерево с обобщённой конфигурацией;
\item \lstinline{Unfolding} обознает шаг символьного вычисления, на котором произошёл шаг вычислений
      и по рассматриваемой на этом шаге конфигурации породились новые конфигурации.
\end{itemize}

%%%%%%%%%%%%%%%%%%%%%%%%%%%%%%%%%%%%%%%%%%%%%%%%%%%%%%%%%%%%%%%%%%%
%%%%%%%%%%%%%%%%%%%%%%%%%%%%%%%%%%%%%%%%%%%%%%%%%%%%%%%%%%%%%%%%%%%
% Описание окружения
%%%%%%%%%%%%%%%%%%%%%%%%%%%%%%%%%%%%%%%%%%%%%%%%%%%%%%%%%%%%%%%%%%%
%%%%%%%%%%%%%%%%%%%%%%%%%%%%%%%%%%%%%%%%%%%%%%%%%%%%%%%%%%%%%%%%%%%

\emph{Окружение} для суперкомпиляции должно сохранять следующие объекты:
\begin{itemize}
\item подстановку, в которой содержатся все накопленные непротиворечивые унификации,
      необходимую в процессе прогонки для проверки новых унификаций;
\item первую свободную семантическую переменную, необходимую для генерации свежих переменных,
      к примеру, при абстракции;
\item определение программы, необходимое для замены вызова на его тело.
\end{itemize}
% \todo{Что-то ещё об этом нужно написать?}

% %%%%%%%%%%%%%%%%%%%%%%%%%%%%%%%%%%%%%%%%%%%%%%%%%%%%%%%%%%%%%%%%%%%
% %%%%%%%%%%%%%%%%%%%%%%%%%%%%%%%%%%%%%%%%%%%%%%%%%%%%%%%%%%%%%%%%%%%
% % Описание шага unfolding'а
% %%%%%%%%%%%%%%%%%%%%%%%%%%%%%%%%%%%%%%%%%%%%%%%%%%%%%%%%%%%%%%%%%%%
% %%%%%%%%%%%%%%%%%%%%%%%%%%%%%%%%%%%%%%%%%%%%%%%%%%%%%%%%%%%%%%%%%%%

% Шаг символьного вычисления по данной конфигурации $C$ порождает
% множество конфигурации $\{ C_1, \dots, C_n \}$, описывающих состояния в которое может перейти
% процесс реального исполнения программы. Классически, шаг символьного
% вычисления соответствует семантике языка, который суперкомпилируется,
% и для \ukanren существует сертифицированная семантика\cite{semanticMK},
% однако описание шага символьного вычисления \ukanren для суперкомпиляции 
% усложнено тем, что реляционные языки не исполняются привычным образом,
% как, к примеру, функциональные программы, и \emph{поиск}, вшитый в семантику,
% не ложится на суперкомпиляцию прямым образом.

% Тогда порождённую конфигурацию можно рассматривать не как непосредственный
% шаг вычисления, но как возможное состояние, в которое может перейти программа.
% Такое состояние появляется путём раскрытия тела одного или нескольких
% конъюнктов конфигурации.

% К примеру, рассмотрим часть программы на \ukanren на рисунке~\ref{fig:unfoldEx}, в котором
% определены какие-то отношения \lstinline{f} и \lstinline{g}.
% \begin{figure}[h!]
% \begin{lstlisting}
% f(a) = f'(a)$\lor$f''(a)
% g(a, b) = g'(a)$\land$g''(b)
% \end{lstlisting}
% \caption{Пример отношений для демонстрации шага символьных вычислений}
% \label{fig:unfoldEx}
% \end{figure}

% Допустим, на шаге суперкомпиляции алгоритм обрабатывает конфигурацию
% \lstinline{f($\text{v}_\text{1}$)$\land$g($\text{v}_\text{1}$, $\text{v}_\text{2}$)}
% хотим сделать шаг символьного вычисления. Рассмотрим несколько способов породить новые конфигурации.
% \begin{itemize}
% \item Если раскроется определение \lstinline{f}, то будут получены новые конфигурации 
%       \lstinline{f'($\text{v}_\text{1}$)$\land$g($\text{v}_\text{1}$, $\text{v}_\text{2}$)} и
%       \lstinline{f''($\text{v}_\text{1}$)$\land$g($\text{v}_\text{1}$, $\text{v}_\text{2}$)}.
% \item Если раскроется определение \lstinline{g}, то будет получена новая конфигурация 
%       \lstinline{f($\text{v}_\text{1}$)$\land$g'($\text{v}_\text{1}$)$\land$g''($\text{v}_\text{2}$)}.
% \item Если раскроются оба определения \lstinline{f} и \lstinline{g}, то будут получены новые конфигурации 
%       \lstinline{f'($\text{v}_\text{1}$)$\land$g($\text{v}_\text{1}$)$\land$g''($\text{v}_\text{2}$)} и
%       \lstinline{f''($\text{v}_\text{1}$)$\land$g($\text{v}_\text{1}$)$\land$g''($\text{v}_\text{2}$)}.
% \end{itemize}

% Последний набор конфигураций --- это полный набор состояний, в которые процесс вычислений может
% прийти. В первых двух наборах, можно отметить, порождённые конфигурации не исключают
% возможные состояния процессов, отображённые в последнем наборе, они могут появится на последующих шагах вычисления,
% если перед этим ветвь исполнения не будет остановлена из-за противоречивой подстановки.

% Таким образом, какой-бы способ развёртывания определений не был бы выбран, он не будет
% исключать состояния, в которые процесс вычисления теоретически может прийти, но выбор
% разных стратегий развёртывания может систематически приводить к разным деревьям процессов,
% а следовательно, приводить к различным эффектам специализации.

Базовой стратегией порождения новых конфигураций выбрана \emph{полная стратегия развёртывания},
пример которой был показан выше, при которой мы заменяем определния всех реляционных вызовов
конфигурации.

В суперкомпиляции, в отличие от методов частичной дедукции, в обобщение включён
шаг обобщения вверх, при котором происходит не подвешивание обобщённой конфигурации
в качестве потомка конфигурации, которая обобщалась, но замена самого родителя на
новую конфигурацию, поддерево же родителя уничтожается. Для определения
необходимости обобщать вверх введём предикат $e_1 \genup e_2$, который
определяет, что $e_1 \strictinst e_2$ и $e_2 \not\strictinst e_1$.
Такое ограничение необходимо из-за того, что суперкомпилятор оперирует
конъюнкциями выражений и делает операции разделения и обобщения вниз
за один шаг с конъюнкциями возможно разной длины, однако для обобщения
вверх необходимо удоставериться, что одни \todo{todo}


% %%%%%%%%%%%%%%%%%%%%%%%%%%%%%%%%%%%%%%%%%%%%%%%%%%%%%%%%%%%%%%%%%%%
% %%%%%%%%%%%%%%%%%%%%%%%%%%%%%%%%%%%%%%%%%%%%%%%%%%%%%%%%%%%%%%%%%%%
% % Описание общего алгоритма суперкомпиляции
% %%%%%%%%%%%%%%%%%%%%%%%%%%%%%%%%%%%%%%%%%%%%%%%%%%%%%%%%%%%%%%%%%%%
% %%%%%%%%%%%%%%%%%%%%%%%%%%%%%%%%%%%%%%%%%%%%%%%%%%%%%%%%%%%%%%%%%%%

% \subsubsection{Обобщённый алгоритм суперкомпиляции}

% На основе введёных выше терминов и операторов можно составить обобщённый алгоритм
% суперкомпиляции, который не затрагивает особенности и трудности реализации на Haskell.
% Обобщённый алгоритм суперкомпиляции на псевдокоде представлен на рисунке~\ref{fig:scalgogen}.

% \begin{figure}[h!]
% \begin{lstlisting}
% drive(env, tree, configuration):
%   if configuration is empty
%   then add(env, tree, success node)
%   else if $\exists$ parent: configuration $\variant$ parent
%   then add(env, tree, renaming node)
%   else if $\exists$ parent: parent $\genup$ configuration
%   then
%      node $\larrow$ generalize(configuration, parent)
%      addUp(env, tree, parent, node)
%   else if $\exists$ parent: parent $\embed^+$ configuration
%   then
%     add(env, tree, abstraction, node)
%     children $\larrow$ generalize(configuration, parent)
%     $\forall \text{child} \in \text{children}:$
%       drive(env, tree, child)
%   else
%     add(env, tree, unfolding node)
%     children $\larrow$ unfold(env, configuration)
%     $\forall \text{child} \in \text{children}:$
%       drive(env, tree, child)
% \end{lstlisting}
% \caption{Обобщённый алгоритм суперкомпиляции.}
% \label{fig:scalgogen}
% \end{figure}


% % Привести пример

% % \begin{figure}[h!]
% % \caption{Пример порождённого дерева для запроса X}
% % \label{fig:treeExample}
% % \end{figure}


%%%%%%%%%%%%%%%%%%%%%%%%%%%%%%%%%%%%%%%%%%%%%%%%%%%%%%%%%%%%%%%%%%%
%%%%%%%%%%%%%%%%%%%%%%%%%%%%%%%%%%%%%%%%%%%%%%%%%%%%%%%%%%%%%%%%%%%
% Описание конкретного алгоритма суперкомпиляции
%%%%%%%%%%%%%%%%%%%%%%%%%%%%%%%%%%%%%%%%%%%%%%%%%%%%%%%%%%%%%%%%%%%
%%%%%%%%%%%%%%%%%%%%%%%%%%%%%%%%%%%%%%%%%%%%%%%%%%%%%%%%%%%%%%%%%%%
\subsubsection{Конкретный алгоритм суперкомпиляции}

Наличие операции обобщения вверх предполагает, что необходимо умение передвигаться по дереву вверх и изменять его. 
Реализация в Haskell этой идеи --- задача крайне нетривиальная. Возможно представлять
деревья в мутабельных массивах, однако при обобщении необходимо удалять целые поддеревья,
что при таком подходе сложная операция.

Классическим способом решения этой проблемы являются \emph{зипперы}\cite{zipper}.
Эта идиома предлагает рассматривать структуру данных как пару из элемента,
на котором установлен фокус, и контекста, который представляется как структура данных
с ``дыркой'', в котором сфокусированный элемент должен находиться.

К примеру, зиппер для списка \lstinline{[1, 2, 3, 4]} при фокусе на 3 представляется
таким образом: \lstinline{(3, ([2, 1, 0], [4, 5, 6]))}.
Тогда перефокусировка вправо или влево на один элемент происходит за константу,
как и замена элемента, для которой достаточно заменить первую компоненту пары.
В то время как, в силу того, что операция взятия элемента в связном списке по индексу
происходит за линейное время от длины списка, взятие элемента слева от 3 также
будет происходить за линейное время, как и, соответственно, модификация списка.

Для деревьев с произвольным количеством детей зиппер может выглядеть
как пара из текущего узла и списка родителей, отсортированного в порядке
близости к узлу (рисунок~\ref{fig:zipper}). 
\begin{figure}[h!]
\begin{lstlisting}[mathescape,language=Haskell,extendedchars=\true,frame=single,basicstyle=\ttfamily]
data Parent = Parent { children :: ListZipper Node }
type TreeZipper = (Node, List Parent)
\end{lstlisting}
\caption{Пример структуры зиппера для деревьев}
\label{fig:zipper}
\end{figure}

Родительский (структура \lstinline{Parent}) cписок детей представлен в виде зиппера (поле \lstinline{children})
для списка, в котором происходит фокус: у непосредственного родителя --- на элемент в фокусе, а у остальных
родителей --- на предыдущего в порядке сортировки.\todo{more?}

При представлении дерева процессов в идиоме зипперов основа алгоритма суперкомпиляции
принимает форму описания действий при смене состояния зиппера.
\todo{TODO}

\subsubsection{Модификации базового алгоритма суперкомпиляции}

\textbf{Поиск узлов для переименования среди всех вычисленных поддеревьев}

В базовом алгоритме суперкомпиляции поиск узлов на которые происходят переименования происходит
среди родителей. Это напрямую соотносится с понятием символьных вычислений: по достижении
узла, которое является переименованием уже встреченного, вычисление переходит на родительский узел.
Однако довольно части встречается, что в разных поддеревьях дерева процессов стречаются одинаковые
конфигурации, поддеревья которых оказываются полностью идентичными. В таком случае, кажется
очевидной оптмизация, при которой мы запоминаем вычисленные поддеревья и в случае,
когда мы встречаем схожу конфигурацию, не вычисляем поддерево заново, добавляя ссылку на него.
\todo{Показать, почему это ничего не сломает}

\textbf{Стратегии развёртывания реляционных вызовов}

Как уже говорилось, разные стратегии развёртывания реляционных вызовов могут привести к разным
эффектам специализации. К примеру, полная стратегия развёртывания, которая была принята за базовую,
приводит к \emph{таплингу} \origin{tupling}\cite{tupling} --- оптмизации, при которой
множество проходов по одной структуре данных заменяется на один проход.

% Может, показать, что оно круто всё строит?

Основной недостаток базового подхода в том, что он для получения всех возможных состояний
производит декартово производение тел вызовов в конъюнкциях, что приводит
к сильному разрастанию дерева процессов и, как следствия, сильно требователен к вычислительным ресурсам.
Вследствие чего реализация новых стратегий развёртывания производится не только в исследовательских,
но и прикладных целях.

Для лёгкой подмены стратегий суперкомпиляции был разработан специальный интерфейс \lstinline{UnfoldableGoal}
(рисунок~\ref{fig:unfoldable}).
\begin{figure}[h!]
\begin{lstlisting}
class Unfoldable a where
   initialize :: Conf $\rarrow$ a
   get        :: a $\rarrow$ Conf
   unfoldStep :: a $\rarrow$ Env $\rarrow$ List (Env, a)
\end{lstlisting}
\caption{Интерфейс для различных стратегий развёртывания.}
\label{fig:unfoldable}
\end{figure}

Предоставляемые интерфейсом функции используются в алгоритме суперкомпиляции следующим образом:
\begin{itemize}
\item \lstinline{initialize} оборачивает конфигурацию в структуру, в которой может содержаться
      вспомогательная информация для процесса развёртывания;
\item \lstinline{get} позволяет получить конфигурацию для применения её к операциям, не зависящим
      от стратегий;
\item \lstinline{unfoldStep} непосредственно проводит шаг вычисления на основе текущей конфигурации
      и её окружения, порождая новые конфигурации с соответствующими им состояниями.
\end{itemize}

В работе рассмотрен и реализован ряд стратегий, описанные ниже.

\begin{itemize}
\item {\bf Последовательная стратегия развёртывания}, при которой отслеживается,
      какой вызов был раскрыт на предыдущем шаге, чтобы на текущем
      раскрыть следующий за ним.\todo{todo}
\item {\bf Нерекурсивная стратегия развёртывания}, при которой в первую очередь
      раскрывается нерекурсивный вызов в конфигурации. Нерекурсивность определяется
      лишь тем, содержит ли определение реляционный вызов самого себя. Более сложный
      анализ структуры функций не мог бы быть использован в силу того, что тогда
      было бы необходимо реализовать класс алгоритмов анализа, что совершенно отдельная задача.

      Ожидается, что при нерекурсивной стратегии развёртывания из конфигураций
      будут как можно быстрее появляться выражения, которые могут быть сокращены
      или вовсе удалены из-за унификации (к примеру, отношения, кодирующие
      таблицы истинности, такие как \rel{and}) или привести к скорой свёрте.
      \todo{todo}.

\item {\bf Стратегия развёртывания вызовов с минимальным количеством ветвлений},
      при которой на каждом шаге вычисления будет появляться минимально возможное количество
      конфигураций, что приведёт к минимальной ветвистости дерева.
\end{itemize}

% \subparagraph{Смешанная стратегия развёртывания}
% \todo{Ещё нужно бы проработать}

\textbf{Частичный или полный отказ от обобщения вверх}

Обобщение вверх приводит к тому, что происходит замена целого поддерева процессов
предка, на которого обобщается конфигурация. Иногда это может приводить к тому, что
теряются аргументы частично известного входа. К примеру, на рисунке~\ref{fig:genup}
представлено дерево процессов, при котором происходит обобщение вверх.
\begin{figure}[h!]
\center
\begin{tikzpicture}[->,node distance=2cm, sibling distance=5cm]
                                                            
  \tikzstyle{conf}=[rectangle,draw, rounded corners=.8ex]

  \node[conf] (root) {\rel{reverse}($a$, $a$)} ;
  \node[conf] (gen) [below of = root] {Generalizer: $\{ v_1 \mapsto a, v_2 \mapsto a \}$};
  \node[conf] (node) [below of = gen] {\rel{reverse}($v_1$, $v_2$)};
  \node (rest)[below of = node] {$\cdots$};
   \path (root) edge (gen)
         (gen) edge (node)
         (node) edge (rest);
  % \path (root) edge node[above left,pos=1] {$\{a \mapsto \text{Zero}\}$} (childLeft)
  %       (root) edge node[above right,pos=1]{$\{a \mapsto \text{Succ}(a_1)\}$}(childRight)
  %       (childLeft) edge (left)
  %       (childRight) edge (childRight2)
  %       (childRight2) edge[bend right=90] (root);
\end{tikzpicture}
\caption{Демонстрация потери информации при обобщении вверх.}
\label{fig:genup}
\end{figure}
\todo{пояснения к примеру}

В случае же, когда обощение происходит на сам корень дерева, теряется эффект протягивания
констант.

\textbf{Обобщение на все вычисленные узлы, не только на родительские}
\todo{Понять, почему это не противоречит методам суперкомпиляции}

\textbf{Расширение языка \ukanren c помощью операции неэквивалентности}

Множество операции в оргинальном \ukanren покрывает все нужды реляционного программирования,
однако на ряде программа оно вычислительно допускает пути исполнения, которые не приводят
к успеху, однако сообщить об этом не представляется возможным.

К примеру, на рисунке~\ref{fig:lookup} изображена операция поиска значения по ключу
в списке пар ключ-значения \rel{lookup}.

\begin{figure}[h!]
\begin{lstlisting}
$\text{lookup}^o$ K L R =
   (K', V) :: L' $\equiv$ L $\land$
   (K' $\equiv$ K $\land$ V $\equiv$ R $\lor$ $\text{lookup}^o$ K L' R)
\end{lstlisting}
\caption{Отношения поиска значения по ключу.}
\label{fig:lookup}
\end{figure}

В соответствии с программой список \lstinline{L} должен иметь в голове пару из ключа и значения \lstinline{(K', V)}
и либо  этот ключ \lstinline{K'} унифицируется с искомым ключом \lstinline{K} и
значение \lstinline{V} --- с результатом \lstinline{R},
либо поиск происходит в хвосте списка \lstinline{L'}. Проблема этой программы в том,
что если унификация \lstinline{(K',V)::L' $\equiv$ L} прошла успешно и был
найден результат, то поиск всё равно продёт во вторую ветку с рекурсивным вызовом и будет
искать значение дальше, хотя по семантике поиска ключа в списке должен вернуться лишь одно значение.
Более того, суперкомпилятору тоже придётся учитывать и, возможно, проводить вычисления,
которые не принесут никакой пользы.

В miniKanren существует операция неэквивалентности $t_1 \not\equiv t_1$, вводящее
ограничение неэквивалентности \origin{disequality contraints}\cite{mkConstr}.
Операция неэквивалентности определяет, что два терма $t_1$ и $t_2$ никогда не должны быть равны,
накладывая ограничения на возможные значения свободных переменных терма.

Расширение синтаксиса \ukanren представлено на рисунке~\ref{fig:syntaxExt}.

\begin{figure}[h!]
\centering
\[\begin{array}{ccll}
\mathcal{G}   & = & \hspace{1cm} \dots & \\
              &   & \hspace{1cm} \mathcal{T_X}\not\equiv\mathcal{T_X} \hspace{2cm} &\mbox{дезунификация} \\
\end{array}\]
\caption{Расширение синтаксиса \ukanren относительно указанного на рисунке~\ref{fig:syntax}.}
\label{fig:syntaxExt}
\end{figure}

Исправленная версия отношения \rel{lookup} представлена на рисунке~\ref{fig:lookupExt}.

\begin{figure}[h!]
\begin{lstlisting}
$\text{lookup}^o$ K L R =
   (K', V) :: L' $\equiv$ L $\land$
   (K' $\equiv$ K $\land$ V $\equiv$ R $\lor$
    K' $\not\equiv$ K $\land$ $\text{lookup}^o$ K L' R)
\end{lstlisting}
\caption{Исправленное отношение поиска значения по ключу.}
\label{fig:lookupExt}
\end{figure}

В такой реализации две по сути исключающие друг друга ветви исполнения будут исключать друг друга
и при вычислении запросов, и при суперкомпиляции.

Для реализации ограничения неэквивалентности вводится новая сущность под названием
``хранилище ограничений'' $\Omega$ \origin{constraints store}, которое используется для проверки
нарушений неэквивалентности. Окружение расширяется хранилищем ограничений, которое затем используется
при унификации и при добавлении новых ограничений.

Тогда нужно ввести следующие модификации в алгоритм унификации конфигурации, который собирает все
операции унификации в конъюнкции перед тем, как добавить её в множество допустимых конфигураций.
\begin{itemize}
\item При встрече операции дезунификации $t_1 \not\equiv t_2$ необходимо произвести следующие действия.
      Применить накопленную подстановку к термам $t_1 \theta = t_1'$ и $t_2 \theta = t_2'$ и 
      унифицировать термы $t_1'$ и $t_2'$. Если получился пустой унификатор, значит, эти термы
      равны и ограничение нарушено. В таком случае суперкомпилятор покинет эту
      ветвь вычислений. Если же термы не унифицируются, значит, никакая подстановка
      в дальнейшем не нарушит ограничение. Иначе необходимо запомнить унификатор в хранилище.
\item При встрече операции унификации $t_1 \equiv t_2$ необходимо получить их унификатор.
      Если его не существует или он пуст, то дополнительных действий производить не нужно.
	  Иначе нужно проверить, не нарушает ли унификатор ограничения неэквивалентности.
\end{itemize}

Указанное расширение было добавлено в библиотеку с реализацией сопуствующих алгоритмов.

Выявление остаточной программы по дереву процессов --- \emph{резидуализация} ---
породит новые опеределения отношений. Больше одного отношения из дерева процессов может
появиться в случае, когда узлы \lstinline{Renaming} указывают на узлы, отличные от корня.
Поэтому первой фазой происходит пометка узлов, задающих таким образом отношения,
а также удаление поддеревьев, у которых все ветви вычисления пришли к неудаче.

Далее происходит обход дерева, во время которого генерируются узлы синтаксического дерева программы
в зависимости от типа текущего узла дерева процессов:
\begin{itemize}
\item \lstinline{Unfoldable} узел приводит к появлению дизъюнкций подпрограмм, которые задают дети этого узла.
      Это обусловлено тем, что при прогонке в этом узле происходит ветвеление вычислений;
\item \lstinline{Abstraction} узел приводит к появлению конъюнкций подпрограмм, которые задают дети этого узла.
	  Это обусловлено тем, что хотя операция обобщения выявляет подконъюнкции из конфигурации и рассматривает их отдельно,
	  оба поддерева, задающиеся этими подконъюнкциями, должны выполнятся в одно и то же время;
\item \lstinline{Generalizer} задаёт обобщающий унификатор, который должен быть добавлен
      перед своим поддеревом;
\item \lstinline{Renaming} формирует вызов реляционного отношения;
\item \lstinline{Success} представляет собой успешное вычисление, предоставляющее непротиворечивую подстановку.
\end{itemize}


\subsection{Модификации суперкомпилятора}

\input{sc/scmod.tex}


\newpage
\section{Тестирование}
\label{sec:testing}

\subsection{Тестовое окружение}

В качестве основной конкретной реализации \ukanren для тестирования
использовался OCanren\footnote{https://github.com/JetBrains-Research/OCanren}\cite{ocanren},
встроенный в OCaml\cite{ocanren}.
Для некоторых тестов для использовался faster-miniKanren\footnote{https://github.com/miniKanren/faster-miniKanren},
версия miniKanren, встроенная в Scheme.

Тесты запускались на платформе: Intel Core i5-6200U CPU, 2.30GHz, DDR4, 12GiB.

Для тестирования суперкомпилятора и его модификаций использовался следующий алгоритм.
\begin{enumerate}
\item На вход предоставляется программа на внутреннем DSL \ukanren библиотеки специализации.
\item Программа и запрос, на который будет происходит специализация, подаются на вход суперкомпилятору.
\item По дереву процессов, порождённому суперкомпилятором, строится остаточная программа.
\item Остаточная программа транслируется в OCanren/faster-miniKanren и
      запускается в заранее подготовленном окружении с тестовыми запросами.
\end{enumerate}


Реализованный суперкомпилятор сравивался с реализацией \forcpd для $\mu$Kanren\footnote{\url{https://github.com/kajigor/uKanren_transformations}},
а также c реализацией \forcpd для Prolog --- системой ECCE\footnote{\url{https://github.com/leuschel/ecce}}.
Для последнего требовалось оттранслировать программу на \ukanren в Prolog, специализировать
её на запрос, далее оттранслировать результирующую программу в OCanren. Все необходимые средства
для этого также предоставлялись указанной библиотекой специализации.

\subsection{Набор тестов}

% Набор мелких тестов на базовую валидацию сгенерированных программ.

\begin{itemize}
\item Отношение сортировки \rel{sort}(list, result). Запросы:
    \begin{itemize}
    \item оптимизация сортировки: \rel{sort}(xs, ys);
    \item генерация отсортированных последовательностей: \rel{sort}(xs, xs).
    \end{itemize}
\item Отношение, проверяющее принадлежность пути графу \rel{isPath}(path, graph, result).
      Специализация \rel{isPath}(path, graph, true). Запросы:
    \begin{itemize}
    \item генерация $n$ произвольных путей в случайном графе;
    \item поиск пути заданного размера в случайном графе: \\ $\text{isPath}^o_s$(p, g)$\land$\rel{length}(p, N).
    \end{itemize}
\item Интерпретатор формул логики высказываний \rel{logint}(formula, subst, result). Запросы:
    \begin{itemize}
    \item поиск $n$ решений заданной формулы;
    \item генерация $n$ формул с подстановке размера $n$.
    \end{itemize}
\item Интерпретатор лямбда-исчисления \rel{lam}(expr, result). Запросы:
	\begin{itemize}
	\item генерация $n$ выражений в нормальной форме \rel{lam}(expr, expr);
	\item генерация $n$ выражений, редуцирующихся к заданному выражению \rel{lam}(expr, E).
	\end{itemize}
\item Проверка типов в просто типизировнном лямбда-исчислении \rel{infer(type, expr)}.
    \begin{itemize}
	\item Поиск $n$ обителей заданного типа.
	\item Генерация выражений, соответствующих заданной спецификации типа и выржения. Специализируется выражение:\\
	      \rel{infer}(type, expr) $\land$ type $\equiv$ TYPE\_SPEC $\land$ expr $\equiv$ EXPR\_SPEC.
    \end{itemize}
\item Интерпретатор простого подмножества Scheme.
   \begin{itemize}
   \item \todo{Интересный тест!}
   \end{itemize}
\end{itemize}

\newpage
\phantomsection
\section*{ЗАКЛЮЧЕНИЕ}
\addcontentsline{toc}{section}{ЗАКЛЮЧЕНИЕ}

В результате проделанной работы был разработан суперкомпилятор для miniKanren,
реализованы его модификации, а также была произведена их апробация.

Реализованный суперкомпилятор показал улучшение производительности на подавляющем
большинстве расмотренных программ относительно исходной программы, а также относительно
реализаций конъюнктивной частичной дедукции,
в иных случаях просадки производительности оказывались несущественными.

Исходный код проекта можно найти на сайте \url{https://github.com/RehMaar/uKanren-spec},
автор принимал участие под учётной записью RehMaar.

Результаты работы были представлены на конференции TEASE-LP'20.


\newpage
\phantomsection
\renewcommand{\refname}{СПИСОК ИСПОЛЬЗОВАННЫХ ИСТОЧНИКОВ}
\addcontentsline{toc}{section}{СПИСОК ИСПОЛЬЗОВАННЫХ ИСТОЧНИКОВ}
\printbibliography

\newpage
\phantomsection
\renewcommand{\refname}{ПРИЛОЖЕНИЕ A}
\addcontentsline{toc}{section}{ПРИЛОЖЕНИЕ А}
\begin{figure}[h!]
\begin{lstlisting}
doubleAppend a b c d =
  fresh (t)
   (appendo a b t $\land$ appendo t c d)
appendo y4 y5 y6 =
  (y4 $\equiv$ [] $\land$ y6 $\equiv$ y5) $\lor$
  fresh (ty t h)
   (y4 $\equiv$ h :: t $\land$
    y6 $\equiv$ h :: ty $\land$
    appendo t y5 ty)
\end{lstlisting}
\caption{Программа для тестирования \lstinline{doubleAppend}}
\end{figure}

\begin{figure}[h!]
\begin{lstlisting}
doubleAppend z1 z2 z3 z4 =
  (z1 $\equiv$ nil () $\land$ app3 z2 z3 z4) $\lor$
  fresh (fE fB fA)
    (z1 $\equiv$ fA :: fB $\land$
    (z4 $\equiv$ fA :: fE) $\land$
    appD fB z2 z3 fE)

appD z1 z2 z3 z4 =
  (z1 $\equiv$ nil ()  $\land$ app3 z2 z3 z4) $\lor$
  fresh (fE fB fA)
   (z1 $\equiv$ fA :: fB $\land$
   (z4 $\equiv$ fA :: fE) $\land$
    appD fB z2 z3 fE)

app3 z1 z2 z3 =
  (z1 $\equiv$ nil () $\land$ (z2 $\equiv$ z3)) $\lor$
  fresh (fD fC fB fA)
    (z1 $\equiv$ fA :: fB $\land$
    (z3 $\equiv$ fA :: fD) $\land$
    app3 fB z2 fD)
\end{lstlisting}
\caption{Специализированная системой ECCE программа \lstinline{doubleAppend}}
\end{figure}

\begin{figure}[h!]
\begin{lstlisting}
doubleAppendo a b c d =
  fresh (x4) (appD a b x4 c d)

appD a b t c d =
   fresh (x7 x6 x5 x10 x9 x8)
     (a $\equiv$ [] $\land$ b $\equiv$ t $\land$
       (t $\equiv$ [] $\land$ c $\equiv$ d $\lor$
       (t $\equiv$ x8 :: x9) $\land$ d $\equiv$ x8 :: x10 $\land$
        appendo x9 c x10
     ) $\lor$
     (a $\equiv$ x5 :: x6 $\land$
      t $\equiv$ x5 :: x7 $\land$
      d $\equiv$ x5 :: x10 $\land$
      appD x6 b x7 c x10) ) 

app a b c =
  fresh (x13 x12 x11)
   (a $\equiv$ [] $\land$ b $\equiv$ c $\lor$
   (a $\equiv$ x11 :: x12 $\land$
    c $\equiv$ x11 :: x13 $\land$
    app x12 b x13))
\end{lstlisting}
\caption{Суперкомпилированная программа \lstinline{doubleAppend}}
\end{figure}

\end{document}
