\section{Тестирование}
\label{sec:testing}

\subsection{Тестовое окружение}

В качестве основной конкретной реализации \ukanren для тестирования
использовался OCanren\footnote{https://github.com/JetBrains-Research/OCanren}\cite{ocanren},
встроенный в OCaml\cite{ocanren}.
Для некоторых тестов для использовался faster-miniKanren\footnote{https://github.com/miniKanren/faster-miniKanren},
версия miniKanren, встроенная в Scheme.

Тесты запускались на платформе: Intel Core i5-6200U CPU, 2.30GHz, DDR4, 12GiB.

Для тестирования суперкомпилятора и его модификаций использовался следующий алгоритм.
\begin{enumerate}
\item На вход предоставляется программа на внутреннем DSL \ukanren библиотеки специализации.
\item Программа и запрос, на который будет происходит специализация, подаются на вход суперкомпилятору.
\item По дереву процессов, порождённому суперкомпилятором, строится остаточная программа.
\item Остаточная программа транслируется в OCanren/faster-miniKanren и
      запускается в заранее подготовленном окружении с тестовыми запросами.
\end{enumerate}


Реализованный суперкомпилятор сравивался с реализацией \forcpd для $\mu$Kanren\footnote{\url{https://github.com/kajigor/uKanren_transformations}},
а также c реализацией \forcpd для Prolog --- системой ECCE\footnote{\url{https://github.com/leuschel/ecce}}.
Для последнего требовалось оттранслировать программу на \ukanren в Prolog, специализировать
её на запрос, далее оттранслировать результирующую программу в OCanren. Все необходимые средства
для этого также предоставлялись указанной библиотекой специализации.

\subsubsection{Набор тестов}

% Набор мелких тестов на базовую валидацию сгенерированных программ.

\begin{itemize}
\item Отношение сортировки \rel{sort}(list, result). Запросы:
    \begin{itemize}
    \item оптимизация сортировки: \rel{sort}(xs, ys);
    \item генерация отсортированных последовательностей: \rel{sort}(xs, xs).
    \end{itemize}
\item Отношение, проверяющее принадлежность пути графу \rel{isPath}(path, graph, result).
      Специализация \rel{isPath}(path, graph, true). Запросы:
    \begin{itemize}
    \item генерация $n$ произвольных путей в случайном графе;
    \item поиск пути заданного размера в случайном графе: \\ $\text{isPath}^o_s$(p, g)$\land$\rel{length}(p, N).
    \end{itemize}
\item Интерпретатор формул логики высказываний \rel{logint}(formula, subst, result). Запросы:
    \begin{itemize}
    \item поиск $n$ решений заданной формулы;
    \item генерация $n$ формул с подстановке размера $n$.
    \end{itemize}
\item Интерпретатор лямбда-исчисления \rel{lam}(expr, result). Запросы:
	\begin{itemize}
	\item генерация $n$ выражений в нормальной форме \rel{lam}(expr, expr);
	\item генерация $n$ выражений, редуцирующихся к заданному выражению \rel{lam}(expr, E).
	\end{itemize}
\item Проверка типов в просто типизировнном лямбда-исчислении \rel{infer(type, expr)}.
    \begin{itemize}
	\item Поиск $n$ обителей заданного типа.
	\item Генерация выражений, соответствующих заданной спецификации типа и выржения. Специализируется выражение:\\
	      \rel{infer}(type, expr) $\land$ type $\equiv$ TYPE\_SPEC $\land$ expr $\equiv$ EXPR\_SPEC.
    \end{itemize}
\item Интерпретатор простого подмножества Scheme.
   \begin{itemize}
   \item \todo{Интересный тест!}
   \end{itemize}
\end{itemize}
