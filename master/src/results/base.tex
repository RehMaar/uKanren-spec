% Для начала разберём, как базовый суперкомпилятор ведёт себя на
% базовом наборе программ, результат которых несложно проанализировать.

% \begin{itemize}
% \item Классическая программа \lstinline{doubleAppend}~\cite{cpd}, которая используется для
%       проверки наличия эффекта дефорестации (\todo{В приложении}).
% %\item Другая классическая программа \lstinline{maxLength}~\cite{cpd}, которая
% %      используется для проверки наличия эффекта таплинга.
% \end{itemize}

Классическая программа для тестирования эффектов специализации называется
\lstinline{doubleAppend}, представленная в приложении \todo{},
в которой происходит конкатенация списков трёх списков.
Во многих бенчмарках~\cite{cpdPract, controlPoly} это программа используется
про проверки эффекта дефорестации.

Рассмотрим дерево процессов на рисунке~\ref{fig:dappTree}, которое порождается применением базового
суперкомпилятора к программе \lstinline{doubleAppend}, причём в
качестве аргументов будут простые свободные переменные, из-за чего
будет просто оптмизироваться сама структура программы.

%\begin{figure}[h!]
%\center
%\begin{tikzpicture}[->,node distance=3cm, sibling distance=6.2cm, level distance=2cm]
%  \tikzstyle{conf}=[rectangle,draw, rounded corners=.8ex,align=center]
%  \node[conf] (root)   at (0, 0)     {{\it Unfolding} \\ \lstinline{doubleAppend(a, b, c, abc)}};
%  \node[conf] (appapp) at (0, -2)    {{\it Unfolding} \\ \lstinline{app(a, b, $\text{v}_0$)$\land$app($\text{v}_0$, c, abc)}};
%  \node[conf] (app1)   at (-0.3, -6) {{\it Unfolding} \\ \lstinline{app($\text{v}_1$, c, $\text{v}_2$)} \\ $\{$ a $\mapsto$ [], b $\mapsto$ $\text{v}_0$, \\ $\text{v}_0 \mapsto \_ : \text{v}_1$, \\ $\ \ $ abc $\mapsto \_ : \text{v}_2$ $\}$};
%  \node[conf] (app2)   at (7, -5.39) {{\it Renaming} \\ \lstinline{app($\text{v}_5$, b, $\text{v}_6$)$\land$app($\text{v}_6$, c, $\text{v}_7$)} \\ $\{$ a $\mapsto \text{h} : \text{v}_5$  $\text{v}_0 \mapsto \text{h} : \text{v}_6 $  abc $\mapsto \text{h} : \text{v}_7$  $\}$};
%  \node[conf] (appS)   at (-4,-6)    {{\it Success} \\ $\{$ a $\mapsto$ [], \\ \ \ b $\mapsto$ [],\\ \ \ $\text{v}_0 \mapsto$ [], \\ \ \ c $\mapsto$ abc $\}$};
%  \node[conf] (app11)  at (2,  -10)  {{\it Renaming} \\ \lstinline{app($\text{v}_3$, c, $\text{v}_4$)} \\ $\{\text{v}_1 \mapsto \_ : \text{v}_3 $ \\ $\ \ \ \ \text{v}_2 \mapsto \_ : \text{v}_4 $  $\}$};
%  \node[conf] (app1S)  at (-2, -9.7) {{\it Success} \\ $\{ \text{v}_1 \mapsto [],$ \\ $\ \ c \ \mapsto \text{v}_2 \}$};
%  \draw[-latex] (root) -- (appapp);
%  \draw[-latex] (appapp) -- (appS.north);
%  \draw[-latex] (appapp) -- (app1);
%  \draw[-latex] (appapp) -- (app2);
%  \draw[-latex] (app1) -- (app1S);
%  \draw[-latex] (app1) -- (app11);
%  \draw[-latex,dashed] (app2) edge [bend right] (root);
%  \draw[-latex,dashed] (app11) edge [bend right] (app1);
%\end{tikzpicture}
%\caption{Дерево процессов для программы \lstinline{doubleAppend}}
%\label{fig:dappTree}
%\end{figure}

В приведённом дереве

