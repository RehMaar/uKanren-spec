{\bf Логическое программирование}~--- это вид декларативного программирования,
основанный на логике предикатов первого порядка в форме дизъюнктов Хорна (то есть дизъюнктов
с только одним положительным литералом),
применяющий принципы логического вывода на основе заданных фактов и правил вывода.
Программа, написанная на логическом языке --- это множество логических формул,
выражающих факты и правила, описывающих определённую область проблем.\cite{logicMJ}

Существует множество языков логического программирования, таких как Prolog, Curry, Mercurry,
однако самые известные --- языки семейства Prolog. Prolog применяется для доказательства
теорем, проектирования баз знаний, создания экспертных систем и искусственного интеллекта.

Prolog построен на \emph{методе резолюций}, который является обобщением метода
``доказательства от противного'', а в частности --- на \emph{линейном} методе
резолюций \origin{Linear resolution with Selection function for Definition clauses}.
При вычислении программы правило резолюции применяется не к случайных дизъюнктам,
а в строго установленном порядке. В случае, когда вычисления дизъюнкта прошло
неудачно, происходит \emph{откат} к прошлому состоянию программы, на котором
выбирался неудавшийся дизъюнкт\cite{logicMJ}.
Помимо этого, Prolog вводит разнообразные синтаксические конструкции с побочными \emph{эффектами},
то есть с действиями, приводящими к изменению \emph{окружения} программы,
к примеру, оперетор отсечения~\origin{cut},
который влияет на способ вычисления программы.

Описанные выше вещи определяют Prolog, однако из-за них теряется свойство \emph{реляционности}.

% 
% Пример программы на Prolog приведён в Листинге~\ref{lst:memberProlog}.
% 
% \begin{lstlisting}[language=Prolog,caption={Проверка принадлежности элемента списку},captionpos=b,label={lst:memberProlog}]
% member(X, [X | T]).
% member(X, [H | T]) :- member(X, T).
% \end{lstlisting}
% 
% В этой программе проверяется принадлежность элемента списку. Есть два возможных
% варианта происходящего: либо элемент равен элементу в голове списка
