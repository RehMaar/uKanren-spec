{\bf Логическое программирование}~--- это вид декларативного программирования,
основанный на логике предикатов первого порядка в форме дизъюнктов Хорна.\cite{logicMJ}
Программа, написанная на логическом языке --- это множество логических формул,
выражающих факты и правила, описывающих определённую область проблем.

Prolog --- один из самых известных языков логического программирования.

\todo{добавить что-нибудь ещё про Prolog}

% Самый известный язык логического программирование --- Prolog.
% 
% {\bf TODO: рассказать про всякие Хорновские клозы и т.п.}
% 
% Пример программы на Prolog приведён в Листинге~\ref{lst:memberProlog}.
% 
% \begin{lstlisting}[language=Prolog,caption={Проверка принадлежности элемента списку},captionpos=b,label={lst:memberProlog}]
% member(X, [X | T]).
% member(X, [H | T]) :- member(X, T).
% \end{lstlisting}
% 
% В этой программе проверяется принадлежность элемента списку. Есть два возможных
% варианта происходящего: либо элемент равен элементу в голове списка
