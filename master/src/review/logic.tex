{\bf Логическое программирование}~--- это вид декларативного программирования,
основанный на формальной логике. Программы представляются в виде
утверждений, представляющихся логическими формулами и описывающих
определённую область проблем.
``Вычисление'' программы в контексте логического программирования производится
в форме \emph{поиска} доказательства утверждений на основе заданных
фактов --- аксиом --- и правил вывода в соответствии с заданной
\emph{стратегией поиска}~\cite{logicMJ}.

Стратегия поиска задаёт,
каким образом происходит обход пространства поиска ответов, и,
как следствие, определяет как, какие и в каком порядке будут находится
ответы. Стратегия поиска, при которой каждый возможный ответ будет со временем
выдан, называется \emph{полной}. Чаще применяются \emph{неполные} стратегии,
поскольку они менее требовательны к вычислительным ресурсам~\cite{currySearch}.

Самые известные языки логического программирования --- языки семейства Prolog.
Prolog применяется для доказательства теорем~\cite{prologTheorem},
проектирования баз знаний, создания экспертных систем~\cite{prologExSys}
и искусственного интеллекта~\cite{prologInt}.
Prolog строится на логике предикатов первого порядка в форме дизъюнктов
Хорна (то есть дизъюнктов только с одним положительным литералом) и
использует \emph{метод резолюций}, основанный на доказательстве от
противного, для решения задач.
Prolog вводит разнообразные синтаксические конструкции с
побочными \emph{эффектами}, то есть с действиями, приводящими к изменению
\emph{окружения} программы, к примеру, оперaтор отсечения~\origin{cut},
который влияет на способ вычисления программы, предотвращая нежелательные
вычисления.
Также он использует стратегию обхода в глубину, что приводит к тому, что
поиск может ``зациклиться'' и никогда не выдать оставшиеся решения,
но, тем не менее, благодаря своим расширениям Prolog
остаётся хорошим решением для задач из своей области применения~\cite{logicMJ}.

% Подобные операторы обычно используются для того, чтобы
% повысить эффективность программ, однако 

% Другие языки логического программровани, такие как Mercury\cite{mercury} или 
% Curry\cite{curry}, являются 

% {\bf Логическое программирование}~--- это вид декларативного программирования,
% основанный на логике предикатов первого порядка в форме дизъюнктов Хорна
% (то есть дизъюнктов
% с только одним положительным литералом),
% применяющий принципы логического вывода на основе заданных фактов и правил.
% Программа, написанная на логическом языке --- это множество логических формул,
% описывающих определённую область проблем.\cite{logicMJ}

% Существует множество языков логического программирования, таких как Prolog, Curry, Mercurry,
% однако самые известные --- языки семейства Prolog. Prolog применяется для доказательства
% теорем, проектирования баз знаний, создания экспертных систем и искусственного интеллекта.

% Prolog построен на \emph{методе резолюций}, который является обобщением метода
% ``доказательства от противного'', а в частности --- на \emph{линейном} методе
% резолюций \origin{Linear resolution with Selection function for Definition clauses}.
% При вычислении программы правило резолюции применяется не к случайных дизъюнктам,
% а в строго установленном порядке. В случае, когда вычисления дизъюнкта прошло
% неудачно, происходит \emph{откат} к прошлому состоянию программы, на котором
% выбирался неудавшийся дизъюнкт\cite{logicMJ}.
% 
% Помимо этого, Prolog вводит разнообразные синтаксические конструкции с побочными \emph{эффектами},
% то есть с действиями, приводящими к изменению \emph{окружения} программы,
% к примеру, оперaтор отсечения~\origin{cut},
% который влияет на способ вычисления программы.

% Это свойства определяют Prolog как язык, однако из-за них теряется свойство \emph{реляционности}.

% % 
% % Пример программы на Prolog приведён в Листинге~\ref{lst:memberProlog}.
% % 
% % \begin{lstlisting}[language=Prolog,caption={Проверка принадлежности элемента списку},captionpos=b,label={lst:memberProlog}]
% % member(X, [X | T]).
% % member(X, [H | T]) :- member(X, T).
% % \end{lstlisting}
% % 
% % В этой программе проверяется принадлежность элемента списку. Есть два возможных
% % варианта происходящего: либо элемент равен элементу в голове списка


