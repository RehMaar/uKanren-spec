{\bf Логическое программирование}~--- это вид декларативного программирования,
основанный на логике предикатов первого порядка в форме дизъюнктов Хорна,
применяющий принципы логического вывода на основе заданных фактов и правил вывода.
Программа, написанная на логическом языке --- это множество логических формул,
выражающих факты и правила, описывающих определённую область проблем.
\cite{logicMJ}

Существует множество языков логического программирования, таких как Prolog, Curry, Mercurry,
однако самые известные --- языки семейства Prolog. Prolog применяется для доказательства
теорем, проектирования баз знаний, создания экспертных систем и искусственного интеллекта.
% Prolog построен на \emph{методе резолюций} (\todo{как-нибудь пояснить}), 
\todo{Сказать, почему мы хотим большего и нам нужно РП}

% 
% Пример программы на Prolog приведён в Листинге~\ref{lst:memberProlog}.
% 
% \begin{lstlisting}[language=Prolog,caption={Проверка принадлежности элемента списку},captionpos=b,label={lst:memberProlog}]
% member(X, [X | T]).
% member(X, [H | T]) :- member(X, T).
% \end{lstlisting}
% 
% В этой программе проверяется принадлежность элемента списку. Есть два возможных
% варианта происходящего: либо элемент равен элементу в голове списка
