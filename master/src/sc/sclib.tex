Реализация суперкомпилятора для \ukanren строилась на основе проекта по специализации miniKanren с помощью конъюнктивной частичной
дедукции\footnote{\url{https://github.com/kajigor/uKanren_transformations/}} на функциональном языке программирования Haskell.
Результаты специализации miniKanren представлены в работе~\cite{lozov}.


Библиотека вводит ряд структур данных и алгоритмов для реализации конъюнктивной частичной дедукции,
однако существует возможность её переиспользования в силу нескольких доводов:
\begin{itemize}
\item схожесть методов частичной дедукции и суперкомпиляции\cite{pdAndDriving},
      из чего следует, что ряд вспомогательных функций можно переиспользовать;
\item алгоритм обобщения конъюнктивной частичной дедукции\cite{cpd}, о котором подробнее будет рассказано позже,
      имеет ряд общих черт с шагами обобщения в суперкомпиляции;
\item библиотека предоставляет возможность преобразования сгенерированной программы
      на miniKanren, при котором удаляются излишние унификации и
      происходит удаление излишних аргументов~\origin{redundant argument filtering},
      что приводит в увличению производительности программы, поскольку унификация операция дорогая.
\end{itemize}

%%%%%%%%%%%%%%%%%%%%%%%%%%%%%%%%%%%%%%%%%%%%%%%%%%%%%%%%%
%%%%%%%%%%%%%%%%%%%%%%%%%%%%%%%%%%%%%%%%%%%%%%%%%%%%%%%%%
% Инстанс
%%%%%%%%%%%%%%%%%%%%%%%%%%%%%%%%%%%%%%%%%%%%%%%%%%%%%%%%%
%%%%%%%%%%%%%%%%%%%%%%%%%%%%%%%%%%%%%%%%%%%%%%%%%%%%%%%%%

Выражение $e_2$ является экземпляром выражения $e_1$, --- $e_2 \prec e_1$ ---
если существует такая подстановка $\theta$, применение которой приравнивает
два выражения $e_1 \theta = e_2$.

%%%%%%%%%%%%%%%%%%%%%%%%%%%%%%%%%%%%%%%%%%%%%%%%%%%%%%%%%
%%%%%%%%%%%%%%%%%%%%%%%%%%%%%%%%%%%%%%%%%%%%%%%%%%%%%%%%%
% Обобщение
%%%%%%%%%%%%%%%%%%%%%%%%%%%%%%%%%%%%%%%%%%%%%%%%%%%%%%%%%
%%%%%%%%%%%%%%%%%%%%%%%%%%%%%%%%%%%%%%%%%%%%%%%%%%%%%%%%%

Алгоритм обобщения основан на понятии \emph{наиболее тесного обобщения}.
\emph{Обобщение} выражения $e_1$ и $e_2$ --- это выражение $e_g$, такое что
$e_g$ 
Наиболее тесное обобщение \origin{most specific generalization} 

%%%%%%%%%%%%%%%%%%%%%%%%%%%%%%%%%%%%%%%%%%%%%%%%%%%%%%%%%
%%%%%%%%%%%%%%%%%%%%%%%%%%%%%%%%%%%%%%%%%%%%%%%%%%%%%%%%%
% Свисток и гомеоморфное вложение
%%%%%%%%%%%%%%%%%%%%%%%%%%%%%%%%%%%%%%%%%%%%%%%%%%%%%%%%%
%%%%%%%%%%%%%%%%%%%%%%%%%%%%%%%%%%%%%%%%%%%%%%%%%%%%%%%%%

В качестве свистка используется отношение \emph{гомеоморфного вложения}.\cite{scGen}
Отношение гомеоморфного вложения $\unlhd$ определено индуктивно:
\begin{itemize}
\item переменные вложены в переменные: $x \embed y$;
\item терм $X$ вложен в конструктор с именем $C$, если он вложен в один из аргументов конструктора:
      $$X \embed C_n(Y_1, \dots, Y_n): \exists i, X \embed Y_i;$$
\item конструкторы с одинаковыми именами состоят в отношении вложения, если в этом отношении
      состоят их аргументы:
      $$C_n(X_1, \dots, X_n) \embed C_n(Y_1, \dots, Y_n): \forall i, X_i \embed Y_i.$$
\end{itemize}

К примеру, выражение $c(b) \embed c(f(b))$, но $f(c(b)) \cancel{\embed} c(f(b))$.
Преимущество использования гомеоморфного вложения, в первую очередь, состоит в том,
что для этого отношения доказано, что на бесконечной последовательности выражений $e_0, e_1, \dots, e_n$
обязательно найдутся такие два индекса $i < j$, что $e_i \embed e_j$, вне зависимости
от того, каким образом последовательность выражений была получена~\cite{scPos}.
Это свойство повзоляет доказать завершаемость алгоритм суперкомпиляции.

Однако отношение гомеоморфного вложения допускает, чтобы термы $f(X, X)$ и $f(X, Y)$
находились в отношении гомеоморфного вложение $f(X, X) \embed f(X, Y)$ в силу того,
что все переменные вкладываются друг в друга. Однако в приведённом примере обобщение
$f(X, X)$ и $f(X, Y)$ не привело бы к более общей конфигурации.

Отношение \emph{строгого} гомеоморфного вложения $\embed^+$ вводит дополнительное
требование, чтобы терм $X$, состоящий в отношении с $Y$, не был \emph{строгим экземпляром} $Y$.~\cite{homeo}
В таком случае отношение $f(X, X) \cancel{\embed}^+ f(X, Y)$, поскольку $f(X, Y)$ является строгим
экземпляром $f(X, X)$ из-за того, что существует подстановка $\{ X = X, Y = X \}$.
