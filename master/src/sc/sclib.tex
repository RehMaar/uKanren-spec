Реализация суперкомпилятора для \ukanren строилась на основе проекта по специализации miniKanren с помощью конъюнктивной частичной
дедукции\footnote{\url{https://github.com/kajigor/uKanren_transformations/}} на функциональном языке программирования Haskell.

\todo{...}

%%%%%%%%%%%%%%%%%%%%%%%%%%%%%%%%%%%%%%%%%%%%%%%%%%%%%%%%%
%%%%%%%%%%%%%%%%%%%%%%%%%%%%%%%%%%%%%%%%%%%%%%%%%%%%%%%%%
% Про инстансы и прочее
%%%%%%%%%%%%%%%%%%%%%%%%%%%%%%%%%%%%%%%%%%%%%%%%%%%%%%%%%
%%%%%%%%%%%%%%%%%%%%%%%%%%%%%%%%%%%%%%%%%%%%%%%%%%%%%%%%%
Терм $t_2$ является \emph{экземпляром} \origin{instance} терма $t_1$, если
существует подстановка $\theta$, такая что $t_1 \theta = t_2$.

$t_2$ является \emph{строгим} экземпляром $t_1$, если  $t_2$ является экземпляром $t_1$ и
$t_1$ не является экземпляром $t_2$.

\todo{Почему нам это важно}

%%%%%%%%%%%%%%%%%%%%%%%%%%%%%%%%%%%%%%%%%%%%%%%%%%%%%%%%%
%%%%%%%%%%%%%%%%%%%%%%%%%%%%%%%%%%%%%%%%%%%%%%%%%%%%%%%%%
% Свисток и гомеоморфное вложение
%%%%%%%%%%%%%%%%%%%%%%%%%%%%%%%%%%%%%%%%%%%%%%%%%%%%%%%%%
%%%%%%%%%%%%%%%%%%%%%%%%%%%%%%%%%%%%%%%%%%%%%%%%%%%%%%%%%

В качестве свистка используется \emph{гомеоморфное вложение}\cite{scGen}.
\todo{Рассказать, почему его хорошо использовать для суперкомпиляции}

Отношение гомеоморфного вложения $\unlhd$ определено индуктивно:
\begin{itemize}
\item переменные вложены в переменные: $x \embed y$;
\item терм $X$ вложен в конструктор с именем $C$, если он вложен в один из аргументов конструктора:
      $$X \embed C_n(Y_1, \dots, Y_n): \exists i, X \embed Y_i;$$
\item конструкторы с одинаковыми именами состоят в отношении вложения, если в этом отношении
      состоят их аргументы:
      $$C_n(X_1, \dots, X_n) \embed C_n(Y_1, \dots, Y_n): \forall i, X_i \embed Y_i.$$
\end{itemize}

Отношение строгого гомеоморфного вложения $\embed^*$ вводит дополнительное
требование, чтобы терм $X$, состоящий в отношении с $Y$, не был \emph{строгим экземпляром} $Y$. 

%%%%%%%%%%%%%%%%%%%%%%%%%%%%%%%%%%%%%%%%%%%%%%%%%%%%%%%%%
%%%%%%%%%%%%%%%%%%%%%%%%%%%%%%%%%%%%%%%%%%%%%%%%%%%%%%%%%
% Обобщение
%%%%%%%%%%%%%%%%%%%%%%%%%%%%%%%%%%%%%%%%%%%%%%%%%%%%%%%%%
%%%%%%%%%%%%%%%%%%%%%%%%%%%%%%%%%%%%%%%%%%%%%%%%%%%%%%%%%

\emph{Наиболее тесное обобщение} \origin{most specific generalization} \todo{вот такое вот оно}.

