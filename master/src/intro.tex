\phantomsection
\section*{ВВЕДЕНИЕ}
\addcontentsline{toc}{section}{ВВЕДЕНИЕ}

% ВВЕДЕНИЕ + ЗАКЛЮЧЕНИЕ -- описание такое, чтобы было понятно о чём работа и
% какие результаты
% В общих чертах вся работа, указать про аналоги

Реляционное программирование~--- это чистая форма логического программирования,
в которой программы представляются как наборы математических отношений~\cite{byrdMK}.
Отношения
не делают различие между выходными и выходными параметрами, из-за чего одно и то же
отношение может решать несколько связанных проблем. К примеру, отношение, задающее
интерпретатор языка, можно использовать не только для вычисления программ по
заданному входу, но и генерировать возможные входные значения по заданному результату
или генерировать сами программы по спецификации входных и выходных значений интерпратетора.

miniKanren~--- это семейство встраиваемых предметно-ориентированных языков программирования~\cite{byrdMK}.
miniKanren был специально сконструирован для того, чтобы поддержать реляционную парадигму,
опираясь на опыт логических языков, таких как языков семейства Prolog~\cite{logicMJ},
Mercury~\cite{mercury} или Curry~\cite{curry}.

Однако реляционная парадигма довольно сложна, потенциал её весьма велик.
Часто наиболее естественный способ записи отношения не является эффективным. В
частности, при задании функциональных отношений как сопоставления выходов
входам, как это наблюдается в примере с интерпретатором, практически
всегда работает медленно.

Специализация --- это техника автоматической оптимизации программ,
при которой на основе программы и её частично известного входа
порождается новая, более оптимальная программа, которая сохраняет семантику
исходной. Для специализации логических языков используются методы частичной дедукции~\cite{advanced},
самый проработанный из которых --- это \cpd\cite{cpd}. Реализация \forcpd для Prolog ECCE показывает
хорошие результаты~\cite{controlPoly}, однако специфика реляционного программирования
и отличие его от логических языков подразумевает возможность разработать более подходящий
метод специализации. Уже существует адаптация \forcpd для miniKanren~\cite{lozov},
однако её результаты нестабильны, и несмотря на то, что в некоторых случаях
производительность программ улучшается, в других -- она может существенно ухудшиться.

Другой подход для специализации --- это суперкомпиляция, 
техника автоматической трансформации и анализа программ,
при которой программа символьно исполняется с сохранением истории вычислений,
на основе которой принимаются решения о трансформациях.
Суперкомпиляция успешно применяется к функциональным и императивным языкам,
однако суперкомпиляция для логических языков не сильно развита. Существуют
работы, посвящённые демонстрации сходства процессов частичной дедукции и суперкомпиляции~\cite{pdAndDriving},
а также суперкомпилятор APROPOS~\cite{apropos}, который, однако, довольно ограничен
в своих возможностях и требует ручного контроля.

В данной работе предлагается способ адаптации и реализации суперкомпилятора для
реляционного языка miniKanren, также рассматриваются его возможные вариации для
повышения производительности реляционных программ и производится экспериментальное
исследование результата.
